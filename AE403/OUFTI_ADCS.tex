% Specify the type of document
\documentclass[10pt]{article}

% Load a number of useful packages
\usepackage{graphicx}
\usepackage{amsmath,amssymb,amsfonts,amsthm}
 \usepackage[margin=1.0in]{geometry}
\usepackage[colorlinks=true]{hyperref}
\usepackage{cite}
\usepackage[caption=false,font=footnotesize]{subfig}

\usepackage{listings}
\usepackage{color} %red, green, blue, yellow, cyan, magenta, black, white
\definecolor{mygreen}{RGB}{28,172,0} % color values Red, Green, Blue
\definecolor{mylilas}{RGB}{170,55,241}


\lstset{language=Matlab,%
    %basicstyle=\color{red},
    breaklines=true,%
    morekeywords={matlab2tikz},
    keywordstyle=\color{blue},%
    morekeywords=[2]{1}, keywordstyle=[2]{\color{black}},
    identifierstyle=\color{black},%
    stringstyle=\color{mylilas},
    commentstyle=\color{mygreen},%
    showstringspaces=false,%without this there will be a symbol in the places where there is a space
    numbers=left,%
    numberstyle={\tiny \color{black}},% size of the numbers
    numbersep=9pt, % this defines how far the numbers are from the text
    emph=[1]{for,end,break},emphstyle=[1]\color{red}, %some words to emphasise
    %emph=[2]{word1,word2}, emphstyle=[2]{style},    
}

\newcommand\scalemath[2]{\scalebox{#1}{\mbox{\ensuremath{\displaystyle #2}}}}


% Say where pictures (if any) will be placed
\graphicspath{{./pictures/}}

% Define title, author, and date
\title{OUFTI 1 \& OUFTI 2 \\
  \large Attitude Determination and Control System}
\author{Emilio R. Gordon}
\date{March 13, 2018}


% Start of document
\begin{document}
% Put the title, author, and date at top of first page
\maketitle

\tableofcontents
\newpage
\section{Introduction }
In 2016, The university of Liege developed the first nanosatellites ever made in
Belgium : OUFTI-1 and OUFTI-2. The satellite is led by students supported by
professors much like our University's SatDev program. OUFTI-1 is a 1U cubesat and is
the first satellite equipped with the amateur radio digital-communication
protocol: the D-STAR technology. Other experiments that will be aboard OUFTI-1 are an
innovative electrical power system as well as high-performance solar cells. The satellite is
not required to point in one specic direction and the Attitude Determination and Control
System (ADCS) subsystem relies on Passive Magnetic Attitude Stabilization (PMAS). The
rst part of this thesis presents this design made of a permanent magnet and hysteretic bars.
The magnet orients the satellite along the Earth's magnetic eld lines and the hysteretic
bars damp its rotational velocities. The inuence of the magnet on the hysteretic bars as
well as the nite elongation of the bars are carefully studied.
OUFTI-2 is the next satellite in the series. Its size is planned to be twice the size of
OUFTI-1 and its main payload will be a radiometer to perform a direct measurement
of the net heating of the Earth. The radiometer is developed in cooperation with the
Royal Meteorological Institute of Belgium and is called Sun-earth IMBAlance (SIMBA)
radiometer. The second part of this master thesis focuses on the feasibility study of an
active attitude control system which satises the requirements of the payload.
\subsection{The OUFTI family}
\subsection{OUFTI-1}
\subsection{OUFTI-2}
\subsection{Attitude Determination and Control System}
\subsubsection{Passive attitude control and OUFTI-1}
\subsubsection{Active attitude control and OUFTI-2}
\section{Mathematical modelling and Coordinate systems}
\section{Passive ADCS of OUFTI-1}
\subsection{Attitude determination of OUFTI-1}
\section{Active Control for Oufti-2}
\subsection{Payload, orbit and requirements}
\subsection{Attitude determination}
\subsubsection{Sun sensors}
\subsubsection{Magnetometers}
\subsubsection{Gyroscopes}
\subsubsection{Star sensors}
\subsection{Attitude control}
\subsubsection{Magnetic torquers}
\subsubsection{Momentum exchange devices}
\subsubsection{Thrusters}
\subsection{Existing hardware for active control }
\subsection{Conclusion}
\section{Simulations of an active attitude control}
\subsection{Application of quaternions to active control}
\subsection{Attitude determination}
\subsection{Torque free motion}
\subsection{Attitude control models}
\subsection{General parameters for the active control simulations}
\subsection{PID controller}
\subsection{The linear quadratic regulator controller}
\subsection{Detumbling controller based on B-dot}
\subsection{Attitude model with full controllability}
\subsection{Attitude model with only magnetic torquers}
\subsubsection{Magnetic torquers with one reaction wheel}
\section{Conclusions}
\subsection{OUFTI-1}
\subsection{OUFTI-2}
\end{document}