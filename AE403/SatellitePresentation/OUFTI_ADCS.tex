% Specify the type of document
\documentclass[12pt]{article}

% Load a number of useful packages
\usepackage{graphicx}
\usepackage{amsmath,amssymb,amsfonts,amsthm}
 \usepackage[margin=1.0in]{geometry}
\usepackage[colorlinks=true]{hyperref}
\usepackage{cite}
\usepackage[caption=false,font=footnotesize]{subfig}

\usepackage{listings}
\usepackage{color} %red, green, blue, yellow, cyan, magenta, black, white

\usepackage{setspace}
\doublespacing

\newcommand\scalemath[2]{\scalebox{#1}{\mbox{\ensuremath{\displaystyle #2}}}}


% Say where pictures (if any) will be placed
\graphicspath{{./pictures/}}

% Define title, author, and date
\title{OUFTI 1\\
  \large Attitude Determination and Control System}
\author{Emilio R. Gordon}
\date{March 13, 2018}


% Start of document
\begin{document}
% Put the title, author, and date at top of first page
\maketitle
{\singlespacing
\tableofcontents}
\newpage
%%%%%%%%%%%%%%%%%%%%%%%%%%%
%%%%%%%%%%%%                                    Section
%%%%%%%%%%%%%%%%%%%%%%%%%%%
\section{Introduction }
In 2016, the University of Leige developed the first nano-satellites ever made in Belgium: OUFTI-1. The satellite is led by students supported by professors much like our University's SatDev program. OUFTI-1 is a 1U cubesat and is the first satellite equipped with the amateur radio digital-communication protocol: D-STAR technology. OUFTI-1 is equipped with other experiments such as an innovative electrical power system and high-performance solar cells.\cite{eoPortal}\cite{SpaceFlight101}
\newline \newline
Due to its mission, OUFTI-1 is not required to point in one specific direction and the Attitude Determination and Control System (ADCS) subsystem relies on Passive Magnetic Attitude Stabilization (PMAS).
\subsection{Attitude Determination and Control System Overview}
The attitude is the orientation of a body-fixed coordinate frame with respect to an external inertial frame. The Attitude Determination and Control System (ADCS) is made of two parts: attitude determination and attitude control.
\newline \newline
Attitude determination refers to the process of measuring and determining a spacecrafts orientation. In the case of OUFTI-1, an accurate attitude determination is not necessary.
\newline \newline
Attitude control refers to the process of orienting the spacecraft to a given direction. The requirements for pointing accuracy, stability, and maneuverability, as well as other mission requirements such as cost, weight, reliability, orbital motion and lifetime are the key parameters which drive the decision of which technique to use. 
\section{Attitude Determination of OUFTI-1}
The goal of the attitude determination of OUFTI-1 is to get an estimation of the in-orbit rotational speed to provide comparisons to the simulations performed. There are many sensors that can determine the attitude, however, none were used aboard OUFTI-1. It was advantageous to use the existing solar panels as analogue sun sensors. The accuracy, though rough, out-weighed the cost of adding any sensors on-board. 
\newline \newline
Recall that OUFTI-1 is equipped with high-performance solar cells. With information on the power received on each side of the satellite, the position of the sun with respect to the satellite can be determined. As a result, the solar panels act as a two-axis sun sensor with a precision of a couple degrees. 
\newline \newline
This approach for attitude determination lacks the ability to retrieve any information on the angular velocity of the satellite around the sun vector. As a result, information retrieved on the angular velocity is perpendicular to the sun vector.
%%%%%%%%%%%%%%%%%%%%%%%%%%%
%%%%%%%%%%%%                                    Section
%%%%%%%%%%%%%%%%%%%%%%%%%%%
\section{Passive Attitude Control of OUFTI-1}
OUFTI-1 does not require high-precision orientation or specific maneuvers. As a result, passive attitude controls were the best solution of attitude control since it is cheap, simple, light-weight and consumes no power. In addition, it does not require attitude determination to work properly.
\newline \newline
There are two main passive control methods for satellites. One approach uses a gravity gradient leading to two stable states with the minor axis (axis with smallest moment of inertia) pointing towards the Earth. The other passive system orients the satellite along the Earth's magnetic field using an onboard magnet. In the case of OUFTI-1, the use of gravity gradient would have complicated the design. As such, OUFTI-1 utilizes passive magnetic attitude stabilization (PMAS).
\newline \newline
Unless some means of damping is provided, the spacecraft will oscillate. This drawback is overcome by adding a damper. The damper will convert oscillation and rotation energy into heat. A good damping method for Low Earth Orbit spacecraft is the use of hysteretic materials known as Hysteresis Damping.
%%%%%%%%%%%%%%%%%%%%%%%%%%%
%%%%%%%%%%%%                             SubSection
%%%%%%%%%%%%%%%%%%%%%%%%%%%
\subsubsection{Requirements}
To summarize, the goal is to align the satellite with the magnetic field direction using permanent magnetics and to stabilize using hysteresis damping. To accomplish this passive magnetic attitude stabilization approach required a fine mathematical simulation of the hysteresis phenomenon and of the satellite dynamics. In addition, strict requirements on the arrangement of the hysteresis materials and the magnet in the satellite were established. As a requirement:
\begin{itemize}
\item The tolerable stabilization time must be no more than one month.
\item After stabilization, the deviation from Earth's magnetic field must not exceed 15$^{\circ}$.
\end{itemize}
%%%%%%%%%%%%%%%%%%%%%%%%%%%
%%%%%%%%%%%%                       SubSubSection
%%%%%%%%%%%%%%%%%%%%%%%%%%%
\subsubsection{Magnet Material Selection}
Magnet selection for OUFTI-1 was  between AlNiCo and Neodymium Iron bore (NdFeB). Material selection relied on two magnetic properties, Remanence and Coercivity. Remanence measures the strength of the magnetic field created by the magnet while Coercivity is the materials resistance to becoming demagnetized. The two materials shared very similar remanence however, AlNiCo-5 have much lower coercivity. AlNiCo-5 were ultimately chosen since they were cheaper and had space flight experience. It's low coercivity can be ignored with proper handling and storage.
%%%%%%%%%%%%%%%%%%%%%%%%%%%
%%%%%%%%%%%%                       SubSubSection
%%%%%%%%%%%%%%%%%%%%%%%%%%%
\subsubsection{Hysteresis Damping}
The hysteresis phenomenon is at the heart of the chosen damping method. Hysteresis is a lag which occurs between the application of a field/force and its subsequent effect. In the magnetic hysteresis, it is the lag which appears between a varying magnetic field "H-field" and the subsequent magnetic induction B. In other words, the magnetic induction depends on the current H-field and the previous magnetic states. The cover of a hysteresis loop is associated with energy loss, and it causes heating. As the satellite rotates in low earth orbit, the hysteresis rods experience the effects of earth magnetic field which varies along its length. The magnetization of the rod will then undergo a hysteresis cycle. Rotational motion will consequently be dissipated as heat, thus providing a  damping term. The energy loss per cycle is proportional to the area of the hysteresis loop.
\newline \newline
The hysteresis bars were ultimately chosen to be made out of Permenorm 5000 H2, a soft magnetic nickel and iron alloy. Basic calculations were computed amongst different materials to find the energy loss per cycle with Permenorm 5000 H2 experiencing $7.75 \frac{J}{m^3}$ in one cycle. In addition, the material has space-flight experience making it more reliable than other material choices. 
\newline \newline
Extensive simulations were computed to figure out the hysteresis material alignment on the craft to factor in damping efficiency and final stabilization. In addition to, the effects of the permanent magnet for attitude control had to be taken into account. The damping efficiency is determined by the hysteresis rods volume and material. since the length of the rods was limited to the size of a cube-sat, many parallel bars were be used. This maintained rod efficiency while also increasing the volume. The parallel rods shouldn't be too close in order to avoid bar mutual demagnetization. If the distance between bars is more than about one third of the length, bar mutual demagnetization can be neglected.
%%%%%%%%%%%%%%%%%%%%%%%%%%%
%%%%%%%%%%%%                       SubSubSection
%%%%%%%%%%%%%%%%%%%%%%%%%%%
\subsubsection{Passive Magnetic Attitude Stabilization (PMAS) Design}
A major concern for the design of PMAS was the magnet's interaction with the hysteric bars as this limits the magnetic moment of the magnet. This concern forces a design with sufficient distances between the magnet and the hysteric materials which is particularly tricky in nano-satellites. Moreover, if a component of the magnet's field is directed along the rod, there is a displacement of the working point in the hysteresis cycle. For that reason, the rods should ideally be as close as possible to the plane that is perpendicular to the magnet axis and passes through its center. Any displacement from this plane leads to a component of the vector field H magnet directed along the rod, which may therefore affect the damping efficiency.
\newline \newline
For the official design of OUFTI-1, the magnetization axis of the permanent magnet must be placed with its magnetic axis parallel to the antennas' plane to encounter the requirements of COM. The exact location is at the center of the edge between the face -X and +Z. For the location of the hysteresis rods, 4 bars were used lining the (-Y, +Z ) and (+Y,+Z) face as shown in the figure below. The results of this configuration allowed for a decrease angular rate of $8\times10^{-8} [\frac{rad}{sec^2}]$ which suggests a stabilization time of about 15 days.
\begin{center}
\includegraphics[scale=0.8]{1.png}
\end{center}
%%%%%%%%%%%%%%%%%%%%%%%%%%%
%%%%%%%%%%%%                             References
%%%%%%%%%%%%%%%%%%%%%%%%%%%
\begin{thebibliography}{9}
\bibitem{eoPortal} 
eoPortal Directory,
\textit{OUFTI-1 (Orbital Utility For Telecommunication Innovation)},
\\\texttt{https://earth.esa.int/web/eoportal/satellite-missions/o/oufti-1}

\bibitem{SpaceFlight101} 
Space Flight 101,
\textit{OUFTI-1},
\\\texttt{http://spaceflight101.com/spacecraft/oufti-1/}

\bibitem{adcs} 
Vincent Francois-Lavet, University of Li�ge, 2009-2010,
\textit{Study of passive and active attitude control systems for the OUFTI nanosatellites},
\\\texttt{http://vincent.francois-l.be/OUFTI\_ADCS\_2010\_05\_31.pdf}

\bibitem{noRe} 
LaLibre.be, BELGA, June 10, 2016;
\textit{Le satellite Oufti-1 ne r�pond plus},
\\\texttt{http://www.lalibre.be/actu/sciences-sante/le-satellite-oufti-1-ne-\\repond-plus-575b02fe35708dcfedac0956}

\bibitem{noRepst} 
Le Vif, BELGA, June 16, 2016;
\textit{Le contact avec le nanosatellite Oufti-1 reste rompu},
\\\texttt{http://www.levif.be/actualite/sciences/le-contact-avec-le-nanosatellite\\-oufti-1-reste-rompu/article-normal-512271.html}

\end{thebibliography}
\end{document}