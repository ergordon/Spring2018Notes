\documentclass[11pt]{article}
\usepackage[utf8]{inputenc} % Para caracteres en espa�ol
\usepackage{amsmath,amsthm,amsfonts,amssymb,amscd}
\usepackage{multirow,booktabs}
\usepackage[table]{xcolor}
\usepackage{fullpage}
\usepackage{lastpage}
\usepackage{enumitem}
\usepackage{multicol}
\usepackage{fancyhdr}
\usepackage{mathrsfs}
\usepackage{pdfpages}
\usepackage{wrapfig}
\usepackage{setspace}
\usepackage{esvect}
\usepackage{calc}
\usepackage{multicol}
\usepackage{cancel}
\usepackage{graphicx}
\graphicspath{ {} }
\usepackage[retainorgcmds]{IEEEtrantools}
\usepackage[margin=3cm]{geometry}
\usepackage{amsmath}
\newlength{\tabcont}
\setlength{\parindent}{0.0in}
\setlength{\parskip}{0.05in}
\usepackage{empheq}
\usepackage{framed}
\usepackage[most]{tcolorbox}
\usepackage{xcolor}
\colorlet{shadecolor}{orange!15}
\parindent 0in
\parskip 12pt
\geometry{margin=1in, headsep=0.25in}
\theoremstyle{definition}
\newtheorem{defn}{Definition}
\newtheorem{reg}{Rule}
\newtheorem{exer}{Exercise}

% Two more packages that make it easy to show MATLAB code
\usepackage[T1]{fontenc}
\usepackage[framed,numbered]{matlab-prettifier}
\lstset{
	style = Matlab-editor,
	basicstyle=\mlttfamily\small,
}

\newtheorem{note}{Note}
\begin{document}
\setcounter{section}{0}

\thispagestyle{empty}

\begin{center}
{\LARGE \bf Homework 5 Extra Credit}\\
{\large AE403 - Spring 2018 \\ Emilio R. Gordon}
\end{center}
\vspace{0mm}
\begin{lstlisting}
clc; clear;

A = -1:.001:1;

% get 2-D mesh for x and y
[a1 a3] = meshgrid(A);	

% check conditions for these values
cond1 = (1+(a1.*3)+(a1.*a3.*1)) > 0;
cond2 = ((1+(a1.*3)+(a1.*a3.*1)).^2 - 16*a1.*a3.*1) > 0;
cond3 = a1.*a3 > 0;
cond4 = a1 > a3;
cond5 = abs(a1) < 1;
cond6 = abs(a3) < 1;

% convert to double for plotting
cond1 = double(cond1);
cond2 = double(cond2);
cond3 = double(cond3);
cond4 = double(cond4);
cond5 = double(cond5);
cond6 = double(cond6);

% set the 0s to NaN so they are not plotted
cond1(cond1 == 0) = NaN;
cond2(cond2 == 0) = NaN;
cond3(cond3 == 0) = NaN;
cond4(cond4 == 0) = NaN;
cond5(cond5 == 0) = NaN;
cond6(cond6 == 0) = NaN;

% multiply the condaces to keep only the common points
cond = cond1.*cond2.*cond3.*cond4.*cond5.*cond6;

s = surf(a1,a3,cond);
axis([-1 1 -1 1])

%Superficial Plotting Code Removed For Space

view(0,90) % change to top view

\end{lstlisting}


\end{document}