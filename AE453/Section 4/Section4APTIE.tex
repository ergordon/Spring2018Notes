\documentclass[11pt]{article}
\usepackage[utf8]{inputenc} % Para caracteres en espa�ol
\usepackage{amsmath,amsthm,amsfonts,amssymb,amscd}
\usepackage{multirow,booktabs}
\usepackage[table]{xcolor}
\usepackage{fullpage}
\usepackage{lastpage}
\usepackage{enumitem}
\usepackage{multicol}
\usepackage{fancyhdr}
\usepackage{mathrsfs}
\usepackage{wrapfig}
\usepackage{setspace}
\usepackage{esvect}
\usepackage{calc}
\usepackage{multicol}
\usepackage{cancel}
\usepackage{graphicx}
\graphicspath{ {pictures/} }
\usepackage[retainorgcmds]{IEEEtrantools}
\usepackage[margin=3cm]{geometry}
\usepackage{amsmath}
\newlength{\tabcont}
\setlength{\parindent}{0.0in}
\setlength{\parskip}{0.05in}
\usepackage{empheq}
\usepackage{framed}
\usepackage{newtxmath}
\usepackage{euscript}
\DeclareMathAlphabet{\mathpzc}{T1}{pzc}{m}{it}
\usepackage[most]{tcolorbox}
\usepackage{xcolor}
\colorlet{shadecolor}{orange!15}
\parindent 0in
\parskip 12pt
\geometry{margin=1in, headsep=0.25in}
\theoremstyle{definition}
\newtheorem{defn}{Definition}
\newtheorem{reg}{Rule}
\newtheorem{exer}{Exercise}
\newtheorem{note}{Note}
\newcommand{\volume}{{\ooalign{\hfil$V$\hfil\cr\kern0.08em--\hfil\cr}}}
\newcommand{\parr}{\mathbin{\|}} % Parralel Symbol
\begin{document}
\setcounter{section}{0}
\setcounter{page}{0}
\setcounter{equation}{0}
%\definecolor{babyblue}{rgb}{0.54, 0.81, 0.94}
\definecolor{babyblueeyes}{rgb}{0.63, 0.79, 0.95}
\definecolor{babyblue}{rgb}{0.69, 0.88, 0.9}

 \pagestyle{fancy}
\fancyhf{}
\rhead{Section 4:  Kinetic Theory}
\rfoot{Page \thepage}
\thispagestyle{empty}


\begin{center}
{\LARGE \bf Section 4:  Kinetic Theory}\\
{\large AE435}\\
Spring 2018
\end{center}
\vspace{5mm}
\section{Pressure, Temperature, and Internal
Energy}
This section draws from Chapter 1 in the Vincenti and Kruger's Introduction Physical Gas Dynamics

In Kinetic Theory we model a gas as a collection of
particles/molecules. 
\begin{center}
\vspace{25mm}
\end{center}
\tableofcontents
\newpage
\subsection{Pressure}
In this section we will model a gas as a collection of particles in a cubical box with sides of length $l$. We will make the following assumptions:
\begin{enumerate}
\item Particles have no internal structure
\item Equilibrium
\item Mean free path ($\lambda > > l$) (no intermolecular collisions)
\end{enumerate}
The assumption of equilibrium makes the other two assumptions
reasonable - at equilibrium, the number of particles in a certain
state must not change. Gas then acts as if the other two assumptions
are true.
\begin{center}
\vspace{70mm}
\textbf{Figure 1}
\end{center}
A particle in the box moves with a velocity:
\begin{equation}
\begin{aligned}
\vv{c} = c_1 \, \hat{x}_1 + c_2 \, \hat{x}_2 +  c_3 \, \hat{x}_3
\end{aligned}
\end{equation}
Such that the speed C is given by
\begin{equation}
\begin{aligned}
c^2 = c_1^2 + c_2^2 +  c_3^2
\end{aligned}
\end{equation}
\newpage
Our analysis begins with pressure so we will be looking at the particles momentum. Consider...
\begin{enumerate}
\item A particle of mass, m, moving in the $\hat{x}$-direction. The particle will have a
momentum, $m \, c_1 \, \hat{x}_1$.
\item If it bounces elastically from the wall at $x_1=l$, the momentum
change is, $2\,m\,c_1\,\hat{x}_1$.
\item The time between collisions is $\frac{2 \, l}{c_1}$.
\item As a result the force (change in total momentum per unit time) on this
wall from 1 particle is:

\begin{equation*}
\begin{aligned}
\vv{F}_l = \frac{\text{Momentum}}{\text{Time}} = \frac{2 \, m \, c_1 \hat{x}_1}{\frac{2 \, l}{c_1}} = \frac{m c_1^2}{l}\, \hat{x}_1
\end{aligned}
\end{equation*}

\item Pressure is the normal component of the force per unit area, so
the pressure on this wall from 1 particle is:

\begin{equation*}
\begin{aligned}
P_l = \frac{\frac{m c_1^2}{l}}{l^2} = \frac{m c_1^2}{l^3}=\frac{m c_1^2}{\volume}
\end{aligned}
\end{equation*}

Note that $l^3$ is the volume, $\volume$, of the box
\item Corresponding equations give the pressure on the other 2 sets
of walls. Pressure on each wall is equal (by equilibrium) so the mean pressure is:

\begin{equation*}
\begin{aligned}
P = \frac{m \, (c_1^2 + c_2^2 +  c_3^2)}{3 \, \volume} = \frac{m \, c^2}{ 3 \, \volume}
\end{aligned}
\end{equation*}

\item For many particles, the pressure becomes:

\begin{equation}
\begin{aligned}
P = \frac{\sum_i \,m_i \, c_i^2}{ 3 \, \volume} \qquad \qquad \Bigg[\frac{N}{m^2} \cdot \frac{m}{m} = \frac{J}{m^3}\Bigg]
\end{aligned}
\end{equation}

Note how the term $,m_i \, c_i^2$ relates back to kinetic energy.
\end{enumerate}
\newpage
\subsection{Translational Energy}
The energy of translation for the ensemble of gas particles is

\begin{equation}
\begin{aligned}
E_{\text{translational}} = \frac{1}{2} \sum_i \, m_i \, c_i^2
\end{aligned}
\end{equation}

So from Equation 3
\begin{equation}
\begin{aligned}
P \, \volume = \frac{2}{3} \, E_{\text{translational}}
\end{aligned}
\end{equation}

Compare this with the empirically-derived equation of state (the
"perfect gas law") from classical thermo:
\begin{framed}
\begin{equation}
\begin{aligned}
P \, \volume = \EuScript{N}\,\hat{R} \,T
\end{aligned}
\end{equation}
Where
\begin{equation*}
\begin{split}
\EuScript{N} &= \text{Number of moles}\\ 
\hat{R} &= \text{Gas constant per mole}\\ 
T &= \text{Temperature}\\
\end{split}
\end{equation*}
\end{framed}
Therefore the energy of translation for the particles in the volume, $\volume$,is:
\begin{equation}
\begin{aligned}
E_{\text{translational}} = \frac{3}{2}\EuScript{N}\,\hat{R} \,T = \frac{1}{2} \sum_i \, m_i \, c_i^2
\end{aligned}
\end{equation}

This can be written in terms of average kinetic energy per particle:
\begin{framed}
\begin{equation*}
\begin{aligned}
\widetilde{e}_{\text{translational}} = \frac{E_{\text{translational}}}{N}
\end{aligned}
\end{equation*}
Where
\begin{equation*}
\begin{split}
N &= \text{Number of Particles Within $\volume$.}
\end{split}
\end{equation*}
\end{framed}
\newpage
Now, with Equation 7, this becomes:
\begin{framed}
\begin{equation}
\begin{aligned}
\widetilde{e}_{\text{translational}} = \frac{3}{2} \, \frac{\EuScript{N}}{N} \, \hat{R}  \, T = \frac{3}{2} \, {\hat{R}}{\hat{N}} \, T
\end{aligned}
\end{equation}
Where
\begin{equation*}
\begin{split}
\hat{N} = \frac{N}{\EuScript{N}} &= 6.022140857 \times 10^{23} \qquad \Bigg[\frac{\text{Particles}}{\text{Mole}}\Bigg]\\
& = \, \text{Avogadro's Number} 
\end{split}
\end{equation*}
\end{framed}
Another way to express Equation 8 is:
\begin{framed}
\begin{equation}
\begin{aligned}
\widetilde{e}_{\text{translational}} = \frac{3}{2} \, k \, T
\end{aligned}
\end{equation}
Where
\begin{equation*}
\begin{split}
k &= 1.3807 \times 10^{-23} \qquad \Bigg[\frac{J}{K}\Bigg]\\
& = \, \text{Boltzmann Constant} 
\end{split}
\end{equation*}
\end{framed}
Now we can rewrite the pressure in terms of the translational
kinetic energy of the particles, using Equation 3, Equation 7 and Equation 9.
\begin{equation}
\begin{aligned}
P = \frac{2}{3} \, \frac{\EuScript{N} \, \hat{N} \, \widetilde{e}_{tr}}{\volume} = \frac{2}{3} \, \frac{\EuScript{N} \, \hat{N}}{\volume} \, \bigg(\frac{3}{2} \, k \, T\bigg) = \frac{\EuScript{N} \, \hat{N} \, k \, T}{\volume}
\end{aligned}
\end{equation}
If we multiply both sides by the total mass within the volume, M,
and rearrange:
\begin{framed}
\begin{equation}
\begin{aligned}
P = \frac{\EuScript{N} \, \hat{N} \, k \, T}{\volume} \, \bigg(\frac{M}{M}\bigg) = \frac{M}{\volume} \, \bigg(\frac{\EuScript{N} \, \hat{N} \, k}{M}\bigg) \, T
\end{aligned}
\end{equation}
Where
\begin{equation*}
\begin{split}
\rho = \frac{M}{\volume} &\qquad \text{Mass Density} \\ \\
\frac{\hat{R}}{M} = R & \qquad  \text{Gas Constant Per Unit Mass} 
\end{split}
\end{equation*}
\end{framed}
Leading us to...
\begin{equation}
\begin{aligned}
P = \rho \, R \, T
\end{aligned}
\end{equation}

\newpage
\end{document}