\documentclass[11pt]{article}
\usepackage[utf8]{inputenc} % Para caracteres en espa�ol
\usepackage{amsmath,amsthm,amsfonts,amssymb,amscd}
\usepackage{multirow,booktabs}
\usepackage[table]{xcolor}
\usepackage{fullpage}
\usepackage{lastpage}
\usepackage{enumitem}
\usepackage{multicol}
\usepackage{fancyhdr}
\usepackage{mathrsfs}
\usepackage{wrapfig}
\usepackage{setspace}
\usepackage{esvect}
\usepackage{calc}
\usepackage{multicol}
\usepackage{cancel}
\usepackage{graphicx}
\graphicspath{ {pictures/} }
\usepackage[retainorgcmds]{IEEEtrantools}
\usepackage[margin=3cm]{geometry}
\usepackage{amsmath}
\newlength{\tabcont}
\setlength{\parindent}{0.0in}
\setlength{\parskip}{0.05in}
\usepackage{empheq}
\usepackage{framed}
\usepackage[most]{tcolorbox}
\usepackage{xcolor}
\colorlet{shadecolor}{orange!15}
\parindent 0in
\parskip 12pt
\geometry{margin=1in, headsep=0.25in}
\theoremstyle{definition}
\newtheorem{defn}{Definition}
\newtheorem{reg}{Rule}
\newtheorem{exer}{Exercise}
\newtheorem{note}{Note}
\newcommand{\volume}{{\ooalign{\hfil$V$\hfil\cr\kern0.08em--\hfil\cr}}}
\newcommand{\parr}{\mathbin{\|}} % Parralel Symbol
\begin{document}
\setcounter{section}{2}
\setcounter{page}{0}
\setcounter{equation}{0}
%\definecolor{babyblue}{rgb}{0.54, 0.81, 0.94}
\definecolor{babyblueeyes}{rgb}{0.63, 0.79, 0.95}
\definecolor{babyblue}{rgb}{0.69, 0.88, 0.9}

 \pagestyle{fancy}
\fancyhf{}
\rhead{Section 4:  Kinetic Theory}
\rfoot{Page \thepage}
\thispagestyle{empty}


\begin{center}
{\LARGE \bf Section 4:  Kinetic Theory}\\
{\large AE435}\\
Spring 2018
\end{center}
\vspace{5mm}
\section{Velocity Distribution Function}

Not all particles move with the same velocity. Also, the
velocity of a particle doesn't remain the same over time.
We need a statistical way to describe this; this the velocity
distribution function.

\begin{center}
\vspace{25mm}
\end{center}
\tableofcontents
\newpage
%%%%%%%%%%%%%%%%%%%%%%%%%%%%%%%%%%%%%%%%%%%%%%%%%%%%%%%%
%%%%%%%%%%%%%%%                           SECTION 1                                     %%%%%%%%%%%%%%%
%%%%%%%%%%%%%%%%%%%%%%%%%%%%%%%%%%%%%%%%%%%%%%%%%%%%%%%%
\subsection{Mass Distribution Function}
To illustrate this idea of distribution function, consider
mass density.

Consider a gas of N particles with mass m in a volume V.

The density is:

If the gas is nonuniform, the density in a differnetial
volume

At a position vector

is

Note that this assumes that is large enough to
contain a large number of particles. Since the particle
mass doesn't change, 

Which we can write as:

The function gives the number of particles per
unit volume as a function of position; a.k.a. a "position
distribution function"

The number of particles in the differential volume
is

So the mass within that differential volume is

We can define a "normalized distribution function" as

So that the total number of particles in is

This normalized distribution function can be interpreted
as a \textbf{probability density function}, that is, the probability
that a given randomly-chosen particle will be in .

Integrating over the entire volume,

So the probability that a particle within V is within V is
100\%.

We can generalize this idea to state:

A distribution function gives the concentration of some quantity per unit "volume" as a function of position in some kind of "space". 
\newpage
%%%%%%%%%%%%%%%%%%%%%%%%%%%%%%%%%%%%%%%%%%%%%%%%%%%%%%%%
%%%%%%%%%%%%%%%                           SECTION 2                                     %%%%%%%%%%%%%%%
%%%%%%%%%%%%%%%%%%%%%%%%%%%%%%%%%%%%%%%%%%%%%%%%%%%%%%%%
\subsection{Velocity Distribution Function}
Now consider particle with velocity

We can define a differential volume in this velocity space

Define local point density such that the number of
particles within velocity range:

is This is the "velocity distribution
function". Like the position distribution function
you have to multiply by a volume to get a real quantity
(the number of particles).

Define a normalized velocity distribution function

This is the probability that a particle will be within the
specific velocity range. The number of particles within 

is

In terms of number density

where the integral over all possible velocities is:

The velocity distribution of particles is important for
determining average quantities. For instance, if we have
some quantity Q that depends on velocity

The mean or expectation value of Q is then:
\newpage
%%%%%%%%%%%%%%%%%%%%%%%%%%%%%%%%%%%%%%%%%%%%%%%%%%%%%%%%
%%%%%%%%%%%%%%%                           SECTION 3                                     %%%%%%%%%%%%%%%
%%%%%%%%%%%%%%%%%%%%%%%%%%%%%%%%%%%%%%%%%%%%%%%%%%%%%%%%
\subsection{Maxwellian Velocity Distribution Function}
A gas at equilibrium has a special velocity distribution
function. Called the Maxwellian velocity distribution.
\begin{itemize}
\item Basic idea is that:
Stationary velocity distribution
\item Collisions deplete and add to population at same rate
\item Thus, no net change.
\end{itemize}
Collision dynamics with simple billiard-ball model leads
to:

This is Maxwellian VDF.

We can break this into components along each axis
The probability that a particle will be within the velocity
space

The Maxwellian VDF looks like this:

Largest probability is at this corresponds to
particles moving perpendicular to the with

We can transform the Maxwellian VDF into a
Maxwellian SPEED distribution function.

Note that the most probable SPEED is not zero.

Also, the mean speed is not the same as the mostprobable-speed.

Finally, the mean squared speed is

Which works out to give the root-mean-square speed.
\end{document}