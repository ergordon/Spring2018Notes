\documentclass[11pt]{article}
\usepackage[utf8]{inputenc} % Para caracteres en espa�ol
\usepackage{amsmath,amsthm,amsfonts,amssymb,amscd}
\usepackage{multirow,booktabs}
\usepackage[table]{xcolor}
\usepackage{fullpage}
\usepackage{lastpage}
\usepackage{enumitem}
\usepackage{multicol}
\usepackage{fancyhdr}
\usepackage{mathrsfs}
\usepackage{wrapfig}
\usepackage{setspace}
\usepackage{esvect}
\usepackage{calc}
\usepackage{multicol}
\usepackage{cancel}
\usepackage{graphicx}
\graphicspath{ {pictures/} }
\usepackage[retainorgcmds]{IEEEtrantools}
\usepackage[margin=3cm]{geometry}
\usepackage{amsmath}
\newlength{\tabcont}
\setlength{\parindent}{0.0in}
\setlength{\parskip}{0.05in}
\usepackage{empheq}
\usepackage{framed}
\usepackage[most]{tcolorbox}
\usepackage{xcolor}
\colorlet{shadecolor}{orange!15}
\parindent 0in
\parskip 12pt
\geometry{margin=1in, headsep=0.25in}
\theoremstyle{definition}
\newtheorem{defn}{Definition}
\newtheorem{reg}{Rule}
\newtheorem{exer}{Exercise}
\newtheorem{note}{Note}
\newcommand{\volume}{{\ooalign{\hfil$V$\hfil\cr\kern0.08em--\hfil\cr}}}
\newcommand{\parr}{\mathbin{\|}} % Parralel Symbol
\begin{document}
\setcounter{section}{2}
\setcounter{page}{0}
\setcounter{equation}{0}
%\definecolor{babyblue}{rgb}{0.54, 0.81, 0.94}
\definecolor{babyblueeyes}{rgb}{0.63, 0.79, 0.95}
\definecolor{babyblue}{rgb}{0.69, 0.88, 0.9}

 \pagestyle{fancy}
\fancyhf{}
\rhead{Section 4:  Kinetic Theory}
\rfoot{Page \thepage}
\thispagestyle{empty}


\begin{center}
{\LARGE \bf Section 4:  Kinetic Theory}\\
{\large AE435}\\
Spring 2018
\end{center}
\vspace{5mm}
\section{Velocity Distribution Function}

Not all particles move with the same velocity. Also, the
velocity of a particle doesn't remain the same over time.
We need a statistical way to describe this; this the velocity
distribution function.
%VDP a statistical way to describe the velocity of the particles in a system. 
\begin{center}
\vspace{25mm}
\end{center}
\tableofcontents
\newpage
%%%%%%%%%%%%%%%%%%%%%%%%%%%%%%%%%%%%%%%%%%%%%%%%%%%%%%%%
%%%%%%%%%%%%%%%                           SECTION 1                                     %%%%%%%%%%%%%%%
%%%%%%%%%%%%%%%%%%%%%%%%%%%%%%%%%%%%%%%%%%%%%%%%%%%%%%%%
\subsection{Mass Distribution Function}
To illustrate this idea of distribution function, consider
mass density.
%A distribution across physicall space. A change in the fensiryt of the fas in physical space. We will mae the analogy of distribution in physical space and distribution in velocity space.
\begin{center}
\vspace{50mm}
%Figure from vincenti and cougar
\textbf{Figure 4}
\end{center}
Consider a gas of N particles with mass m in a volume V.

The density is:
\begin{equation}
\begin{aligned}
a
\end{aligned}
\end{equation}
If the gas is nonuniform, the density in a differnetial %not the same everywhere
volume
\begin{equation*}
\begin{aligned}
a
\end{aligned}
\end{equation*}
At a position vector
\begin{equation*}
\begin{aligned}
a
\end{aligned}
\end{equation*}
is %then the density at that location is:
\begin{equation}
\begin{aligned}
a
\end{aligned}
\end{equation}
%we still have the assumption that it still contains a large number of particles
Note that this assumes that is large enough to
contain a large number of particles. Since the particle
mass doesn't change, 
\begin{equation*}
\begin{aligned}
a
\end{aligned}
\end{equation*}
Which we can write as:
\begin{equation}
\begin{aligned}
a
\end{aligned}
\end{equation}
%the mass densit can be different depending on where you are at in physical space.
The function gives the number of particles per
unit volume as a function of position; a.k.a. a "position
distribution function"

The number of particles in the differential volume
is
\begin{equation*}
\begin{aligned}
a
\end{aligned}
\end{equation*}
So the mass within that differential volume is
%position distribution function is an absolte, it gives you number at a specific loacation. Typically we work on it in a normalized form making it more of a probability density funcition
\begin{equation*}
\begin{aligned}
a
\end{aligned}
\end{equation*}
We can define a "normalized distribution function" as
\begin{equation}
\begin{aligned}
a
\end{aligned}
\end{equation}%the percent of particals at position xi
So that the total number of particles in is
\begin{equation}
\begin{aligned}
a
\end{aligned}
\end{equation}
This normalized distribution function can be interpreted
as a \textbf{probability density function}, that is, the probability
that a given randomly-chosen particle will be in . %probability that a particles will be at that location

Integrating over the entire volume,
\begin{equation*}
\begin{aligned}
a
\end{aligned}
\end{equation*}
%if we integrate over all volume, the probability of particle 
So the probability that a particle within V is within V is
100\%.

We can generalize this idea to state:

A distribution function gives the concentration of some quantity per unit "volume" as a function of position in some kind of "space". 
% Here it was physical space, the number of particles in a given spacial volume within physical space. We can make the analogy now into a velocity distribution function. The number of particles that are within a volume and velocity space. How are the particles distributed across velocity space.
\newpage
%%%%%%%%%%%%%%%%%%%%%%%%%%%%%%%%%%%%%%%%%%%%%%%%%%%%%%%%
%%%%%%%%%%%%%%%                           SECTION 2                                     %%%%%%%%%%%%%%%
%%%%%%%%%%%%%%%%%%%%%%%%%%%%%%%%%%%%%%%%%%%%%%%%%%%%%%%%
\subsection{Velocity Distribution Function}
Now consider particle with velocity

We can define a differential volume in this velocity space

Define local point density such that the number of
particles within velocity range: %the number of particles in the velocity grange
\begin{equation*}
\begin{aligned}
a
\end{aligned}
\end{equation*}
we would write that as . This is the "velocity distribution
function". Like the position distribution function
you have to multiply by a volume to get a real quantity
(the number of particles). %tells us how particles are distributed across velocity space, much like the position distributiuon function, we have to multiply by the volume to get the overall number of particls

Define a normalized velocity distribution function
\begin{equation}
\begin{aligned}
a
\end{aligned}
\end{equation}
%the fraction of particles that will be in the dvc elemet
This is the probability that a particle will be within the
specific velocity range. The number of particles within is
\begin{equation}
\begin{aligned}
a
\end{aligned}
\end{equation}
In terms of number density

where the integral over all possible velocities is:
\begin{equation*}
\begin{aligned}
a
\end{aligned}
\end{equation*}
%if you look over all of belocity sace, you will find N particles since that is the amount you have in the system. In other words, if we integrate over all of velocity space, the fraction of particles we will find is 1, 100%.
\begin{equation}
\begin{aligned}
a
\end{aligned}
\end{equation}
The velocity distribution of particles is important for
determining average quantities. For instance, if we have
some quantity Q that depends on velocity

The mean or expectation value of Q is then:
\begin{equation}
\begin{aligned}
a
\end{aligned}
\end{equation}
%integrating over the distribution function. If Q is a velocity, then it is called taking the moment of the velocity distribution function.
%for exaample if you wre to do it of <C> would be the first moment. Useul for finding the average collision rate, weher Q is the collision cross section which is tied with the particle velcoty
%We want to find the average of some propertiy that we are interested in finding, velocity, energy, over the entier system.
\newpage
%%%%%%%%%%%%%%%%%%%%%%%%%%%%%%%%%%%%%%%%%%%%%%%%%%%%%%%%
%%%%%%%%%%%%%%%                           SECTION 3                                     %%%%%%%%%%%%%%%
%%%%%%%%%%%%%%%%%%%%%%%%%%%%%%%%%%%%%%%%%%%%%%%%%%%%%%%%
\subsection{Maxwellian Velocity Distribution Function}
%distribution functions have many different forms but this is special. Maxwellian velocity is for the case of equillibirum. Not often the case for EP systems since plasmas are rearely in equillibrium.
A gas at equilibrium has a special velocity distribution
function. Called the Maxwellian velocity distribution.

Basic idea is that:
\begin{itemize}
\item Stationary velocity distribution
\item Collisions deplete and add to population at same rate % acollision that creates a certain outcome, cancels out with another collision. 
\item Thus, no net change.
\end{itemize}
Collision dynamics with simple billiard-ball model leads
to:
%mass is conserved in collsion. Subscript means its maxwellian
\begin{shaded}
This is Maxwellian VDF.
\begin{equation}
\begin{aligned}
a
\end{aligned}
\end{equation}
where
\begin{equation*}
\begin{aligned}
a
\end{aligned}
\end{equation*}
\end{shaded}

We can break this into components along each axis

\begin{equation*}
\begin{aligned}
a
\end{aligned}
\end{equation*}

The probability that a particle will be within the velocity
space
\begin{equation*}
\begin{aligned}
a
\end{aligned}
\end{equation*}
%which is ust the pdf of the x compoment, y and z

The Maxwellian VDF looks like this:
%PLOT: smaller curve is the vdf, the negative x axis sint shown but its symmetric. particles can have negative velocity (this is why we have a symmetric negative componenet. The largest probability is at C1 =0, particles moving perpendicular to the X- axis with ci x1 = 0.  
Largest probability is at this corresponds to
particles moving perpendicular to the with
%The larger curve is the pseed distributation
We can transform the Maxwellian VDF into a
Maxwellian SPEED distribution function.%the speed distribution is not symmetric can cannot be on the negative x axis. speed cannot be negative
\begin{equation}
\begin{aligned}
a
\end{aligned}
\end{equation}
Note that the most probable SPEED is not zero.
%this figure calls out some important points, Peak is the most prbably. cbar is the average speed, cbar^2 is the root mean square mean.
\begin{equation}
\begin{aligned}
a
\end{aligned}
\end{equation}
Also, the mean speed is not the same as the mostprobable-speed. % always the peak.... 
\begin{equation}
\begin{aligned}
a
\end{aligned}
\end{equation}%13% higher
Finally, the mean squared speed is
\begin{equation*}
\begin{aligned}
a
\end{aligned}
\end{equation*}% so now we want the avrage speed squared so we integrae over function and find that this is^
Which works out to give the root-mean-square speed.
\begin{equation}
\begin{aligned}
a
\end{aligned}
\end{equation}%22% higher
%What this is saying is particles have a mean sqrt speed of this only if our gas is in equillibrium. 


%other distributions...in ep, specfically ion, you have thermal electrons insiced the discharge chamber. What would thermal ions look like (thermal = equillibirum). What kind of distribution would they have....maxwellion.....They would have those curves. In addition ot thermal ions, you have cathode that emitting high energy electrons resulting a somewhat bump on tail distribution due to the high energy particles. The electrons inside this discharge chamber are not in equillibrium and cannot be thourouglt described by the maxwellian distribution.
\end{document}