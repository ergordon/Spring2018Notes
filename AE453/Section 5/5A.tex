\documentclass[11pt]{article}
\usepackage[utf8]{inputenc} % Para caracteres en espa�ol
\usepackage{amsmath,amsthm,amsfonts,amssymb,amscd}
\usepackage{multirow,booktabs}
\usepackage[table]{xcolor}
\usepackage{fullpage}
\usepackage{lastpage}
\usepackage{enumitem}
\usepackage{multicol}
\usepackage{fancyhdr}
\usepackage{mathrsfs}
\usepackage{wrapfig}
\usepackage{setspace}
\usepackage{esvect}
\usepackage{calc}
\usepackage{multicol}
\usepackage{cancel}
\usepackage{graphicx}
\graphicspath{ {pictures/} }
\usepackage[retainorgcmds]{IEEEtrantools}
\usepackage[margin=3cm]{geometry}
\usepackage{amsmath}
\newlength{\tabcont}
\setlength{\parindent}{0.0in}
\setlength{\parskip}{0.05in}
\usepackage{empheq}
\usepackage{framed}
\usepackage[most]{tcolorbox}
\usepackage{xcolor}
\colorlet{shadecolor}{orange!15}
\parindent 0in
\parskip 12pt
\geometry{margin=1in, headsep=0.25in}
\theoremstyle{definition}
\newtheorem{defn}{Definition}
\newtheorem{reg}{Rule}
\newtheorem{exer}{Exercise}
\newtheorem{note}{Note}
\newcommand{\volume}{{\ooalign{\hfil$V$\hfil\cr\kern0.08em--\hfil\cr}}}
\newcommand{\parr}{\mathbin{\|}} % Parralel Symbol
\begin{document}
\setcounter{section}{2}
\setcounter{page}{0}
\setcounter{equation}{0}
%\definecolor{babyblue}{rgb}{0.54, 0.81, 0.94}
\definecolor{babyblueeyes}{rgb}{0.63, 0.79, 0.95}
\definecolor{babyblue}{rgb}{0.69, 0.88, 0.9}

 \pagestyle{fancy}
\fancyhf{}
\rhead{Section 4:  Kinetic Theory}
\rfoot{Page \thepage}
\thispagestyle{empty}


\begin{center}
{\LARGE \bf Section 4:  Kinetic Theory}\\
{\large AE435}\\
Spring 2018
\end{center}
\vspace{5mm}
\section{Velocity Distribution Function}

Not all particles move with the same velocity. Also, the
velocity of a particle doesn't remain the same over time.
We need a statistical way to describe this; this the velocity
distribution function.
%VDP a statistical way to describe the velocity of the particles in a system. 
\begin{center}
\vspace{25mm}
\end{center}
\tableofcontents
\newpage
%%%%%%%%%%%%%%%%%%%%%%%%%%%%%%%%%%%%%%%%%%%%%%%%%%%%%%%%
%%%%%%%%%%%%%%%                           SECTION 1                                     %%%%%%%%%%%%%%%
%%%%%%%%%%%%%%%%%%%%%%%%%%%%%%%%%%%%%%%%%%%%%%%%%%%%%%%%
\subsection{Mass Distribution Function}
%we can classivy collisions by the type and then by the 
%causes signinficant changes on wether or not a collision can happen>WE are aobut to see alot of experimental data. Cross section is emperically determined parameter.

%You fire electros into that target fass with some speed or velicty that know,.Out in the perimeter, you will ahve some detector to measure for whatever those scattered particles are. You can use a detectro to find how many xenons you get or even find the ...

%cross section is the target size, the size the target sees as it moves though a collection of particles. Different tyoes fo collisions have different target seze, energy dependent. 

%differential corss sections.... now we are scattered to this angle, the differential cross section tells us cross section into a sepcigic area. 

%I would ahve the the same terminalogy for all the collsions.

%The nitial flux...bohr radius is a_0

%recognize 5.6 is mean free path from before but becofe we used atomic cross section, now we are doing bohr radius or find Q off of a data chart.

%total cross section is...

%differention cross section form , q(theta)

%figure from jahn

% total cross section for that typce of collisoin. 

%we rarely use most proabble speed.
%We typically use the thermal speed. tuw e can use any of them, it would not change equations much...
\end{document}