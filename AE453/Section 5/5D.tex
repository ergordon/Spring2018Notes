\documentclass[11pt]{article}
\usepackage[utf8]{inputenc} % Para caracteres en espa�ol
\usepackage{amsmath,amsthm,amsfonts,amssymb,amscd}
\usepackage{multirow,booktabs}
\usepackage[table]{xcolor}
\usepackage{fullpage}
\usepackage{lastpage}
\usepackage{enumitem}
\usepackage{multicol}
\usepackage{fancyhdr}
\usepackage{mathrsfs}
\usepackage{wrapfig}
\usepackage{setspace}
\usepackage{esvect}
\usepackage{calc}
\usepackage{multicol}
\usepackage{cancel}
\usepackage{graphicx}
\graphicspath{ {pictures/} }
\usepackage[retainorgcmds]{IEEEtrantools}
\usepackage[margin=3cm]{geometry}
\usepackage{amsmath}
\newlength{\tabcont}
\setlength{\parindent}{0.0in}
\setlength{\parskip}{0.05in}
\usepackage{empheq}
\usepackage{framed}
\usepackage{newtxmath}
\usepackage{euscript}
\DeclareMathAlphabet{\mathpzc}{T1}{pzc}{m}{it}
\usepackage[most]{tcolorbox}
\usepackage{xcolor}
\colorlet{shadecolor}{orange!15}
\parindent 0in
\parskip 12pt
\geometry{margin=1in, headsep=0.25in}
\theoremstyle{definition}
\newtheorem{defn}{Definition}
\newtheorem{reg}{Rule}
\newtheorem{exer}{Exercise}
\newtheorem{note}{Note}
\newcommand{\volume}{{\ooalign{\hfil$V$\hfil\cr\kern0.08em--\hfil\cr}}}
\newcommand{\parr}{\mathbin{\|}} % Parralel Symbol
\begin{document}
\setcounter{section}{3} %Section before the section you want. I want section 1 I put 0
\setcounter{page}{24} %page number you want to be the first page
\setcounter{equation}{24} %equation before the equation you want I want equation 2 I put 1
%\definecolor{babyblue}{rgb}{0.54, 0.81, 0.94}
\definecolor{babyblueeyes}{rgb}{0.63, 0.79, 0.95}
\definecolor{babyblue}{rgb}{0.69, 0.88, 0.9}

 \pagestyle{fancy}
 
\fancyhf{}
\rhead{Section 5:  Collisions}
\rfoot{Page \thepage}
\thispagestyle{empty}


\begin{center}
{\LARGE \bf Section 5:  Collisions}\\
{\large AE435}\\
Spring 2018
\end{center}

\vspace{5mm}
\section{Electron-Electron and Ion-Ion Collisions}
\vspace{25mm}
\tableofcontents
\newpage
\subsection{Elastic}
Electron-Electron collisions in electric propulsion plasmas are kinda important but we never really analyze. These types of collisions lead to thermalization (equilibrium). 
\begin{itemize}
\item Same physics as electron-ion Coulomb collisions, but with repelling forces.  Same cross-section shape.  \item Masses are equal so energy transfer is much more efficient.
\item This is the most important process for thermalization within a species 
\end{itemize}

When we say thermalization, we mean reaching equilibrium, a thermalized state. The mechanism by which they thermalize is via collisions.

In typical electric propulsion plasma, $\nu_e$ the electron electron collision frequency is much larger than the ion ion collision frequency.

 \begin{equation*}
 \begin{aligned}
 \nu_{e-e} > > \nu_{i-i}
 \end{aligned}
 \end{equation*}
 
This is because electrons weigh a lot less and move a lot faster. In general, electrons will be thermalized and will have lots more collisions while the ions may not ever reach a thermalized state since they are heavier and slower.
\subsection{Inelastic}
Would require lots of energy for two like charges to get that close together. Particles would have to get really close to get to second-ionization state.

\subsection{Radiation}
Masses are equal so no net acceleration, and thus no Bremsstrahlung 
\vfill
\begin{center}
\textbf{Overall, elastic electron-electron collisions are most dominate in electric propulsion systems.}
\end{center}

\end{document}