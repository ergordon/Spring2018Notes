\documentclass[11pt]{article}
\usepackage[utf8]{inputenc} % Para caracteres en espa�ol
\usepackage{amsmath,amsthm,amsfonts,amssymb,amscd}
\usepackage{multirow,booktabs}
\usepackage[table]{xcolor}
\usepackage{fullpage}
\usepackage{lastpage}
\usepackage{enumitem}
\usepackage{multicol}
\usepackage{fancyhdr}
\usepackage{mathrsfs}
\usepackage{wrapfig}
\usepackage{setspace}
\usepackage{esvect}
\usepackage{calc}
\usepackage{multicol}
\usepackage{cancel}
\usepackage{graphicx}
\graphicspath{ {pictures/} }
\usepackage[retainorgcmds]{IEEEtrantools}
\usepackage[margin=3cm]{geometry}
\usepackage{amsmath}
\newlength{\tabcont}
\setlength{\parindent}{0.0in}
\setlength{\parskip}{0.05in}
\usepackage{empheq}
\usepackage{framed}
\usepackage{newtxmath}
\usepackage{euscript}
\DeclareMathAlphabet{\mathpzc}{T1}{pzc}{m}{it}
\usepackage[most]{tcolorbox}
\usepackage{xcolor}
\colorlet{shadecolor}{orange!15}
\parindent 0in
\parskip 12pt
\geometry{margin=1in, headsep=0.25in}
\theoremstyle{definition}
\newtheorem{defn}{Definition}
\newtheorem{reg}{Rule}
\newtheorem{exer}{Exercise}
\newtheorem{note}{Note}
\newcommand{\volume}{{\ooalign{\hfil$V$\hfil\cr\kern0.08em--\hfil\cr}}}
\newcommand{\parr}{\mathbin{\|}} % Parralel Symbol
\begin{document}
\setcounter{section}{-1}
\setcounter{page}{0}
\setcounter{equation}{0}
%\definecolor{babyblue}{rgb}{0.54, 0.81, 0.94}
\definecolor{babyblueeyes}{rgb}{0.63, 0.79, 0.95}
\definecolor{babyblue}{rgb}{0.69, 0.88, 0.9}

 \pagestyle{fancy}
\fancyhf{}
\rhead{Section 6:  Ionization}
\rfoot{Page \thepage}
\thispagestyle{empty}


\begin{center}
{\LARGE \bf Section 6:  Ionization}\\
{\large AE435}\\
Spring 2018
\end{center}
\vspace{5mm}
\section{Intro}
We can specify the state of a plasma via the species continuity equation:
\begin{shaded}
\textbf{State of Plasma}
 \begin{equation*}
 \begin{aligned}
 a
 \end{aligned}
 \end{equation*}
Where
 \begin{equation*}
 \begin{aligned}
 a & \text{ is the rate of change of species i in the volume} \\ 
 a & \text{ is the convection rate for species i into the volume} \\ 
 a & \text{ is the net generation of species i  in the volume due to process j} \\ 
 \end{aligned}
 \end{equation*}
 \end{shaded}
 For instance, we could use
 \begin{itemize}
\item Electron-impact ionization
\item Radiative recombination
\end{itemize}
And so on through all the processes of the last chapter, Chapter V.
 
We could, in principle solve N equations for N species if we knew all the reaction rate constants             	.   For very simple, low-density plasmas, we use the corona model to do just that.   For higher-density plasmas we have to use a collisional-radiative model, which is more general but requires lots of number-crunching and assumptions.
 
For really high density, the gas reaches equilibrium, and ionization modeling becomes much simpler.


\end{document}