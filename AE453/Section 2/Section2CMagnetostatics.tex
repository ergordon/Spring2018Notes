\documentclass[11pt]{article}
\usepackage[utf8]{inputenc} % Para caracteres en espa�ol
\usepackage{amsmath,amsthm,amsfonts,amssymb,amscd}
\usepackage{multirow,booktabs}
\usepackage[table]{xcolor}
\usepackage{fullpage}
\usepackage{lastpage}
\usepackage{enumitem}
\usepackage{multicol}
\usepackage{fancyhdr}
\usepackage{mathrsfs}
\usepackage{wrapfig}
\usepackage{setspace}
\usepackage{esvect}
\usepackage{calc}
\usepackage{multicol}
\usepackage{cancel}
\usepackage{graphicx}
\graphicspath{ {pictures/} }
\usepackage[retainorgcmds]{IEEEtrantools}
\usepackage[margin=3cm]{geometry}
\usepackage{amsmath}
\newlength{\tabcont}
\setlength{\parindent}{0.0in}
\setlength{\parskip}{0.05in}
\usepackage{empheq}
\usepackage{framed}
\usepackage[most]{tcolorbox}
\usepackage{xcolor}
\colorlet{shadecolor}{orange!15}
\parindent 0in
\parskip 12pt
\geometry{margin=1in, headsep=0.25in}
\theoremstyle{definition}
\newtheorem{defn}{Definition}
\newtheorem{reg}{Rule}
\newtheorem{exer}{Exercise}
\newtheorem{note}{Note}
\newcommand{\volume}{{\ooalign{\hfil$V$\hfil\cr\kern0.08em--\hfil\cr}}}
\newcommand{\parr}{\mathbin{\|}} % Parralel Symbol
\begin{document}
\setcounter{section}{1}
\setcounter{page}{11}
\setcounter{equation}{22}

 \pagestyle{fancy}
\fancyhf{}
\rhead{Section 2: Electromagnetics}
\rfoot{Page \thepage}
\thispagestyle{empty}


\begin{center}
{\LARGE \bf Section 2: Electromagnetics}\\
{\large AE435}\\
Spring 2018
\end{center}
In this section, we will review the basics of charge, electricity, magnetism, and Maxwell equations.
\vspace{25mm}
\section{Electrostatics with Dielectric Media}
\vspace{5mm}
\tableofcontents
\newpage
\subsection{Polarization}
Many molecules have a positive end and a negative end.
\newline
\newline
Water, H$_{2}$O, for example is a "polar molecule".
\begin{center}
\vspace{30mm}
\textbf{Figure 4}
\end{center}
Oxygen gets slightly more than its share of the electron cloud while hydrogen gets less.
\newline
\newline
Water is an example of a "polar molecule". It is always polarized, even without any electric field present. But even non-polar atoms/molecules can become polarized in an electric field. Polarized, meaning a redistribution of the electron cloud creating an asymmetric charge distribution that aligns with the electric field.
\newline
\newline
We model this by thinking of each polarized atom/molecule as a dipole.
\begin{center}
\vfill
\textbf{Figure 5}
\end{center}
\newpage
Polarization (how skewed the electron cloud is) depends on $\vv{E}$. However, part of $\vv{E}$ is produced by polarization. In addition, the redistribution of charge in a dielectric (insulator) can affect the external charge distribution, which changes $\vv{E}$. In short, we have a non-linear system!
\newline
\newline
A dipole is two equal and opposite charges, $\pm q$, separated by small distance, $\vv{l}$.
\begin{center}
\vfill
\textbf{Figure 6}
\end{center}
We start by defining the dipole potential field.
\begin{shaded}
\textbf{Dipole Potential Field} \newline
\begin{equation}
\phi(\vv{r}) =  \frac{q}{4 \pi \epsilon_0} \frac{(\vv{r}-\vv{r}')}{|\vv{r}-\vv{r}'|^3} \cdot \vv{l}
\end{equation}
\newline
Note: We ignore the higher order terms in $\vv{l}$.
\end{shaded}
This equation is exact for a point dipole, where $|\vv{l}| \rightarrow 0$ as $q \rightarrow 0$. All nonlinear terms vanish for a point dipole, which has no net charge or spatial extent, and has a constant dipole moment.
\newline
\begin{equation}
\vv{p} = \lim_{l \rightarrow 0} q \vv{l} = \text{constant}
\end{equation}
\newline
For dielectric media, we can generalize this to
\newline
\begin{equation*}
\Delta \vv{p} = \int_{\volume} \vv{r} \, \mathrm{d}q
\end{equation*}
\newline
which is the dipole moment of charge distribution in a small volume.
\newpage
Usually, it is more convenient to work with the dipole moment per unit volume
\begin{shaded}
\textbf{Polarization of the Dielectric} \newline
\begin{equation}
\vv{P} = \lim_{\Delta \volume \rightarrow 0} \frac{\Delta \vv{p}}{\Delta \volume}
\end{equation}
\newline
This is a point property, $\vv{P} (\vv{r})$ unit [C/m3], called the Polarization of the dielectric. It's a vector with direction defined by charge separation.
\end{shaded}
\begin{center}
\vfill
\textbf{Figure 7}
\end{center}
Another way to think of this is through physical molecules, rather than just point dipoles:
\newline
\begin{equation}
\begin{align}
\vv{p_m} = \text{Polarization of a Molecule} \\
\vv{P} = N \vv{p_m} \quad \big[\frac{\#}{m^3}\big]
\end{align}
\end{equation}
\newline
where $N$ is the number density of molecules in a dielectric.
\newpage
\subsection{Surface and Volume Charge Density}
Charge displacement due to induced dipoles results in a net surface charge density
\begin{shaded}
\textbf{Net Surface Charge Density} \newline
\begin{equation}
\sigma_p = \vv{P} \cdot \hat{n} \qquad \bigg[\frac{c}{m^2}\bigg]
\end{equation}
\end{shaded}
and a net volume charge density.
\begin{shaded}
\textbf{Net Volume Charge Density} \newline
\begin{equation}
\rho_p = -\nabla \cdot \vv{P} \qquad \bigg[\frac{c}{m^3}\bigg]
\end{equation}
\end{shaded}
\begin{center}
\textbf{The rest of this section will be spend proving Equation 27 and Equation 28.}
\end{center}
\newline
\newline
Consider the boundary of a dielectric shown below. Assume all positive charges in the slab move a displacement vector $\vv{S}$ in response to an applied electric field, $\vv{E}$, and that negative charges are stationary.
\begin{center}
\vfill
\textbf{Figure 8}
\end{center}
\newpage
We find that the differential volume defined by displacement $\vv{S}$ and projected area $\mathrm{d}\vv{A} = \hat{n} \, \mathrm{d}\vv{A}$ is:
\begin{equation*}
\mathrm{d}\volume = \vv{s} \cdot \mathrm{d}\vv{A}
\end{equation*}
If N is the number of positive charges per unit volume in the medium, the charge crossing $\mathrm{d}\vv{A}$ is:
\begin{equation*}
\mathrm{d}q = N q \vv{s} \cdot \mathrm{d}\vv{A}
\end{equation*}
\newline
Recall polarization per molecule (Equation 24) $\qquad \, \, \, \, \vv{p_m} = q \, \vv{s}$
\newline
\newline
And polarization per unit volume (Equation 26) $\qquad \vv{P} = N \, \vv{p_m}$
\newline
\newline
\newline
So charge crossing $\mathrm{d}\vv{A}$,
\newline
\begin{equation*}
\mathrm{d}q = \vv{P} \cdot \mathrm{d}\vv{A}
\end{equation*}
\begin{framed}
Then the \textbf{Net Surface Charge} is the charge per unit area:
\newline
\begin{equation}
\frac{\mathrm{d}q}{\mathrm{d}A} = \sigma_p = \vv{P} \cdot \hat{n}
\end{equation}
\newline
So whenever $\vv{P}$ and $\hat{n}$ are in the same direction, we get a net positive (+) surface charge. Whenever $\vv{P}$ and $\hat{n}$ are in the opposite direction, we get a net negative (-) surface charge.
\end{framed}
\newline
\newline
Similar for Volume Charge Density
\newline
Again, the charge flow out of any surface $\mathrm{d}A$ with normal $\hat{n}$ is:
\newline
\begin{equation*}
\mathrm{d}q = \vv{P} \cdot \mathrm{d}\vv{A} = \vv{P} \cdot \hat{n} \, \mathrm{d}A
\end{equation*}
So the net flow out of a differential volume equals the total flux integrated over the surface,
\newline
\begin{equation*}
\int_{S} \vv{P} \cdot \mathrm{d}\vv{A} = \int_{S} \frac{\mathrm{d}q}{\mathrm{d}A} \, \mathrm{d}A
\end{equation*}
\newline
No charge is created or destroyed, so the differential loss of charge within $\mathrm{d} \volume $ is:
\newline
\begin{equation*}
- \int_{S} \mathrm{d}q = \int_{\volume} \rho_p \, \mathrm{d}\volume = - \int_{S} \vv{P} \cdot \mathrm{d}\vv{A} = - \int_{\volume} \nabla \cdot \vv{P} \, \mathrm{d}\volume
\end{equation*}
\newline
Thus, similar to Gauss' Law, you get a differential form valid for any arbitrary volume.
\begin{framed}
Then the \textbf{Net Volume Charge} is the charge per unit volume:
\newline
\begin{equation}
\rho_p = -\nabla \cdot \vv{P}
\end{equation}
\newline
So, a negative gradient in polarization creates a positive volumetric charge. Meanwhile, a positive gradient in polarization creates a negative volumetric charge.
\end{framed}
\newpage
\subsection{Gauss' Law for Dielectrics}
Starting with the free-space Gauss's law, we add dielectric charges
\newline
\begin{shaded}
\textbf{Gauss' Law with Dielectric Charges} \newline
\begin{equation}
\begin{aligned}
\oint_{S} \vv{E} \cdot \mathrm{d}\vv{A} = \int_{\volume} \nabla \cdot \vv{E} \, \mathrm{d} \volume = \frac{\sum q}{\epsilon_0} = \frac{1}{\epsilon_0} \int_{\volume} (\rho_f + \rho_p) \, \mathrm{d}\volume
\end{aligned}
\end{equation}
Where:
\begin{equation*}
\begin{split}
\rho_f &= \text{volume charge density of free charges} \\
\rho_p &= \text{volume charge density of charge due to dielectric polarization} \\
\end{split}
\end{equation*}
\end{shaded}
\newline
The differential form is then (compare with 2.13):
\newline
\begin{equation}
\begin{aligned}
\nabla \cdot \vv{E} = \frac{1}{\epsilon_0} \, \big( \rho_f + \rho_p \big)
\end{aligned}
\end{equation}
\newline
Total electric field is due to free charges AND polarization "charges" .
\newline
\newline
E is due to ALL charges, but usually don't care about, $\rho_p$.
\newline
\newline
What we'd like is an equation in terms of free charge as a source
term. Recall (2.30), so that
\newline
\begin{equation*}
\begin{aligned}
\rho_p =& - \nabla \cdot \vv{P} \\ \\
\nabla \cdot \vv{E} =& \frac{1}{\epsilon_0} \big(\rho_f - \nabla \cdot \vv{P} \big)
\end{aligned}
\end{equation*}
Then:
\begin{equation}
\begin{aligned}
\nabla \cdot \big( \epsilon_0 \vv{E} + \vv{P}\big) = \rho_f
\end{aligned}
\end{equation}
\newline
Define Electric Displacement
\newline
\begin{equation}
\begin{aligned}
\vv{D} = \epsilon_0 \vv{E} + \vv{P}
\end{aligned}
\end{equation}
\newline
The above equation (2.33) becomes:
\newline
\begin{equation}
\begin{aligned}
\nabla \cdot \vv{D} = \rho_f
\end{aligned}
\end{equation}
\newline
Which is the most common form of Poisson's equation seen in Maxwell equations.
\newline
\newline
Free charge is the source of displacement . In integral form,
\newline
\begin{equation}
\begin{aligned}
\oint_S \vv{D} \cdot \, \mathrm{d}\vv{A} = \int_{\volume} \rho_f \, \mathrm{d}\volume
\end{aligned}
\end{equation}
\newline
In contrast, total charge is the source of the Electric Field.
\newline
\begin{equation*}
\begin{aligned}
\big(\rho_f + \rho_p\big)
\end{aligned}
\end{equation*}
\newpage
\subsection{Susceptibility, Permittivity, \& Dielectric Constant}
Material properties are characterized by constitutive relations (a relationship between two physical quantities that is specific to a material). For isotropic materials, 
\begin{shaded}
\textbf{Equation Name} \newline
\begin{equation}
\vv{P} = \chi \big(E\big) \, \vv{E}
\end{equation}
Where:
\begin{equation*}
\begin{split}
\chi &= \text{Electrical Susceptibility of the Material} \\
\end{split}
\end{equation*}
\end{shaded}
Can define permittivity as:
\newline
\begin{equation}
\begin{aligned}
\epsilon (E) = \epsilon_0 + \chi(E)
\end{aligned}
\end{equation}
\newline
So that
\newline
\begin{equation}
\begin{aligned}
\vv{D} = \epsilon(E) \, \vv{E}
\end{aligned}
\end{equation}
\newline
More generally, for time-varying fields, permittivity is a function of both the wave number and frequency.
\newline
\begin{equation*}
\begin{aligned}
\vv{D}(\vv{K},w) = \epsilon (\vv{K}, w) \, \vv{E} (\vv{K}, w)
\end{aligned}
\end{equation*}
\newline
Often $\chi$ and $\epsilon$ are independent of $\vv{E}$, these materials are called linear dielectrics.
\newline
\begin{equation}
\begin{aligned}
\vv{P} = \chi \vv{E} \\ \\
\vv{D} = \epsilon \vv{E}
\end{aligned}
\end{equation}
\newline
For these, can also specify material's electrical behavior by dielectric constant.
\newline
\begin{equation}
\begin{aligned}
\epsilon = K \, \epsilon_0
\end{aligned}
\end{equation}
So
\begin{framed}
\textbf{Relative Permitivity}
\begin{equation}
\begin{aligned}
\epsilon_r = K = \frac{\epsilon}{\epsilon_0} = 1 + \frac{\chi}{\epsilon_0}
\end{aligned}
\end{equation}
\end{framed}
Most materials have a defined dielectric constant and strength. The table below list a few example materials. 
\newpage
$$
\vspace{5mm}

\begin{note}\textbf{The dielectric constant, K, is $\approx 1$ for most gasses.}
\end{note}
\begin{center}
 \begin{tabular}{||c c c||} 
 \hline
 & Dielectric Constant & Dielectric Strength\\ 
 Material & K & E$_{\text{max}}$ $[\frac{V}{m}]$ \\[0.5ex] 
 \hline\hline
 Al$_2$O (Alumina) & 4.5 & 6 $\times 10^6$  \\ 
 \hline
 Glass & 5-10 & 9 $\times 10^6$ \\
 \hline
 Nylon & 3.5 & 19 $\times 10^6$ \\
 \hline
 Ethanol (0$^\circ$ C) & 28.4 & - \\
 \hline
 H$_2$O (0$^\circ$ C) & 87.8 & - \\ 
  \hline
  Air, 1atm & 1.00059 & 3 $\times 10^6$ \\
  \hline
CO$_2$, 1atm & 1.000985 & - \\
\hline
BN (Boron Nitride) & 4.3 & 2.4 $\times 10^6$ \\[1ex] 
 \hline
\end{tabular}
\end{center}
To analyze problems involving dielectrics, you need a table of Dielectric Constants:
\begin{enumerate}
\item Boundary Conditions
\item Governing Equations
\end{enumerate}
\newpage
\subsection{Boundary Conditions at Interfaces}
Consider a generalized surface with external surface charge density, $\sigma_e$. This does NOT include polarization charge, $\sigma_p$.
\begin{center}
\vfill
\textbf{Figure 9}
\end{center}
Where 
\begin{equation*}
\begin{aligned}
\vv{D}_1 = \text{The Electric Displacement just below the surface}\\
\vv{D}_2 = \text{The Electric Displacement just above the surface}
\end{aligned}
\end{equation*}
Note that $\vv{D}_2$ changes only because of the effect of the external electric field. The change is related to the charge on the surface.
\newline
\newline
From the figure, it can be shown that...
\begin{framed}
The difference in the vertical components of the Electric Displacement above and below the generalized surface is equal to the external surface charge density.
\begin{equation}
\begin{aligned}
\big(D_{\perp}\big)_1 - \big(D_{\perp}\big)_2 = \sigma_e
\end{aligned}
\end{equation}
The perpendicular component of Electric Displacement changes proportional to $\sigma$
\end{framed}
If we apply Gauss' law to a small volume slightly above and slightly below the surface and then loot at the field around the surface, we can derive Equation 44.
\begin{framed}
The horizontal components of Electric Displacement above and below the generalized surface are equal. Since we did not include Polarization, $\vv{P}$, our equation for electric displacement, $\vv{D} = \epsilon_0 \vv{E} + \vv{P}$, becomes $\vv{D} = \epsilon_0 \vv{E}$ and so...
\begin{equation}
\begin{aligned}
\big(E_{\parr}\big)_1 = \big(E_{\parr}\big)_2
\end{aligned}
\end{equation}
The parallel components of the Electric Field, $\vv{E}$, do not change above and below the surface.
\end{framed}
\newpage
\subsection{Poisson \& Laplace's Eqns. for Dielectrics}
When we include dielectric media, the divergence equation is: (Derived earlier in Equation 35)
\newline
\begin{equation}
\begin{aligned}
\nabla \cdot \vv{D} = \rho_f
\end{aligned}
\end{equation}
\newline
where $\rho_f$ is the volume density of free charges, and the displacement is
\newline
\begin{equation}
\begin{aligned}
\vv{D} = \epsilon \, \vv{E}
\end{aligned}
\end{equation}
\newline
So, in dielectric media, the divergence of the electric field is given by:
\newline
\begin{equation}
\begin{aligned}
\nabla \cdot \vv{E} = \frac{\rho_f}{\epsilon}
\end{aligned}
\end{equation}
\newline
Which is just what we had for Gauss' Law except now it includes dielectrics. But we still have electrostatic fields, so can put in terms of scalar potential as:
\newline
\begin{equation}
\begin{aligned}
\nabla^2 \phi = -\frac{\rho_f}{\epsilon}
\end{aligned}
\end{equation}
\newline
which is Poisson's Equation for dielectrics.
\newline
\newline
\newline
In most cases, there is no free charge distributed through the dielectric, it concentrates either on the surface or (more rarely) concentrates in clumps within the dielectric. As a result, you can usually use Laplace's Equation for Dielectrics
\newline
\begin{equation}
\begin{aligned}
\nabla^2 \phi = 0
\end{aligned}
\end{equation}
\newline
in the body of a dielectric.
\end{document}