\documentclass[11pt]{article}
\usepackage[utf8]{inputenc} % Para caracteres en espa�ol
\usepackage{amsmath,amsthm,amsfonts,amssymb,amscd}
\usepackage{multirow,booktabs}
\usepackage[table]{xcolor}
\usepackage{fullpage}
\usepackage{lastpage}
\usepackage{enumitem}
\usepackage{multicol}
\usepackage{fancyhdr}
\usepackage{mathrsfs}
\usepackage{wrapfig}
\usepackage{setspace}
\usepackage{esvect}
\usepackage{calc}
\usepackage{multicol}
\usepackage{cancel}
\usepackage{graphicx}
\graphicspath{ {pictures/} }
\usepackage[retainorgcmds]{IEEEtrantools}
\usepackage[margin=3cm]{geometry}
\usepackage{amsmath}
\newlength{\tabcont}
\setlength{\parindent}{0.0in}
\setlength{\parskip}{0.05in}
\usepackage{empheq}
\usepackage{framed}
\usepackage[most]{tcolorbox}
\usepackage{xcolor}
\colorlet{shadecolor}{orange!15}
\parindent 0in
\parskip 12pt
\geometry{margin=1in, headsep=0.25in}
\theoremstyle{definition}
\newtheorem{defn}{Definition}
\newtheorem{reg}{Rule}
\newtheorem{exer}{Exercise}
\newtheorem{note}{Note}
\newcommand{\volume}{{\ooalign{\hfil$V$\hfil\cr\kern0.08em--\hfil\cr}}}
\newcommand{\parr}{\mathbin{\|}} % Parralel Symbol
\begin{document}
\setcounter{section}{4}
\setcounter{page}{48}
\setcounter{equation}{89}

 \pagestyle{fancy}
\fancyhf{}
\rhead{Section 2: Electromagnetics}
\rfoot{Page \thepage}
\thispagestyle{empty}


\begin{center}
{\LARGE \bf Section 2: Electromagnetics}\\
{\large AE435}\\
Spring 2018
\end{center}
In this section, we will review the basics of charge, electricity, magnetism, and Maxwell equations.
\vspace{5mm}
\section{Electromagnetic Induction}
\begin{center}
\vspace{25mm}
\end{center}
\tableofcontents
\newpage
\subsection{Faraday's Induction Law}
From electrostatics, we have the identical statements (Equation 14)
\newline
\begin{equation*}
\begin{aligned}
\nabla \times \vv{E} = 0
\end{aligned}
\end{equation*}
\newline
Or in integral form
\newline
\begin{equation*}
\begin{aligned}
\oint \vv{E} \cdot \mathrm{d}\vv{l} = 0
\end{aligned}
\end{equation*}
\newline
When we have time-varying B-fields, though, we find experimentally that
\newline
\begin{equation}
\begin{aligned}
\oint \vv{E} \cdot \mathrm{d}\vv{l} = \varepsilon_{mf}
\end{aligned}
\end{equation}
\newline
This EMF, "electromotive force", is the potential difference that gives rise to a current and is due to a time-varying magnetic flux.

Consider a conductor element $\mathrm{d}\vv{l}$ moving at velocity $\vv{v}$ in a
B-field, oriented OUT of the page. Electrons in $\mathrm{d}\vv{l}$ feel the
Lorentz force which we first seen in Equation 64
\newline
\begin{equation}
\begin{aligned}
\vv{F}_m = -q_e \, \vv{v} \times \vv{B}
\end{aligned}
\end{equation}
\newline
Thus $\vv{v} \times \vv{B}$ acts just like an applied electric field E. If the
circuit were closed, then the current would flow as if a battery were
supplying a voltage:
\newline
\begin{equation}
\begin{aligned}
\varepsilon_{mf} = \oint_c (\vv{v} \times \vv{B}) \cdot \mathrm{d}\vv{l}
\end{aligned}
\end{equation}
\newline
Use vector identity to rearrange terms
\newline
\begin{equation*}
\begin{aligned}
\varepsilon_{mf} = - \oint_c \vv{B} \cdot (\vv{v} \times \mathrm{d}\vv{l}) = - \frac{\partial }{\partial t} \int_S \vv{B} \cdot \hat{n} \, \mathrm{d}A
\end{aligned}
\end{equation*}
\newline
And note $(\vv{v} \times \mathrm{d}\vv{l}) = \hat{n} \, \frac{\mathrm{d}A}{\mathrm{d}t}$ that is the rate of change of the projected area enclosed by the circuit. Thus, using the definition
of magnetic flux (Equation 88):
\newline
\begin{shaded}
\textbf{Faraday's Induction Law}
\begin{equation}
\begin{aligned}
\varepsilon_{mf} = - \frac{\partial }{\partial t} \int_S \vv{B} \cdot \hat{n} \, \mathrm{d}A = - \frac{\partial \Phi}{\partial t}
\end{aligned}
\end{equation}
\end{shaded}
\newline
\newpage
\begin{framed}
\textbf{Example:} Let's pull a rectangular loop of width l out of a uniform
magnetic field $\vv{B}$ with a velocity $\vv{v}$.
\begin{center}
\vspace{40mm}
\textbf{Figure 22}
\end{center}
For this simple case, the EMF
\newline
\begin{equation*}
\begin{aligned}
\varepsilon_{mf} = \oint_c (\vv{v} \times \vv{B}) \cdot \mathrm{d}\vv{l}
\end{aligned}
\end{equation*}
\newline
Which becomes
\newline
\begin{equation*}
\begin{aligned}
\varepsilon_{mf} = v \, B \, l
\end{aligned}
\end{equation*}
\newline
Since the vertical component is the only non-zero part. Given a loop resistance, R, the loop current is: (V = IR)
\newline
\begin{equation*}
\begin{aligned}
J = \frac{v \, B \, l}{R}
\end{aligned}
\end{equation*}
\newline
(we will assume J is small such that the self-field is negligible).
\newline
Another way to consider this is as a rate of change of magnetic flux:
\newline
\begin{equation*}
\begin{aligned}
 \frac{\partial \Phi}{\partial t} = - B \frac{\mathrm{d}A}{\mathrm{d}t} = -Bvl
\end{aligned}
\end{equation*}
\newline
So again, $\varepsilon_{mf} = v \, B \, l$
In this case, the restoring force: $\vv{F} = q \vv{v} \times \vv{B} = J \vv{l} \times \vv{B}$
Has a magnitude
\newline
\begin{equation*}
\begin{aligned}
\vv{F} = J\,v\,B = \frac{B^2vl^2}{R} \qquad \text{Force to Pull Loop}
\end{aligned}
\end{equation*}
\newline
Thus the power required to pull the loop out is
\newline
\begin{equation*}
\begin{aligned}
P = F \cdot v = \frac{B^2v^2l^2}{R}
\end{aligned}
\end{equation*}
\newline
EP application: Tether Propulsion
\end{framed}
\newpage
\subsection{Faraday's Law}
Consider a time-varying $\vv{B}$
\begin{center}
\vspace{40mm}
\textbf{Figure 23}
\end{center}
The induced EMF in a closed loop (Equation 90)
\newline
\begin{equation*}
\begin{aligned}
\varepsilon_{mf} = \oint \vv{E} \cdot \mathrm{d}\vv{l}
\end{aligned}
\end{equation*}
\newline
is, via Stoke's theorem (Equation 73)
\newline
\begin{equation*}
\begin{aligned}
\varepsilon_{mf} = \oint \vv{E} \cdot \mathrm{d}\vv{l} = \int_S \nabla \times \vv{E} \cdot \hat{n} \,  \mathrm{d} A
\end{aligned}
\end{equation*}
\newline
Applying Faraday's induction law (Equation 93):
\newline
\begin{equation*}
\begin{aligned}
\varepsilon_{mf} = - \frac{\partial}{\partial t} \int_S \vv{B} \cdot \hat{n}  \, \mathrm{d}A = -\int_S  \frac{\partial \vv{B}}{\partial t} \cdot \hat{n} \,  \mathrm{d}A
\end{aligned}
\end{equation*}
\newline
We see that:
\newline
\begin{equation*}
\begin{aligned}
- \frac{\partial}{\partial t} \int_S \vv{B} \cdot \hat{n} \,  \mathrm{d}A = \int_S \nabla \times \vv{E} \cdot \hat{n} \,  \mathrm{d} A
\end{aligned}
\end{equation*}
\newline
Equal terms within the integral for arbitrary integration surfaces means that:
\begin{shaded}
\textbf{Generalized Form of Faraday's Law}
\begin{equation}
\begin{aligned}
\nabla \times \vv{E} = -\frac{\partial \vv{B}}{\partial t}
\end{aligned}
\end{equation}
\end{shaded}
\newpage
\subsection{Lenz's Law}
\textbf{Lenz's Law:} The direction of the induced current is such that its magnetic field opposes the change in flux.
\begin{itemize}
\item If $\frac{\mathrm{d}B}{\mathrm{d}t}$ is negative (i.e., B decreasing with time)
\item The resulting positive Emf induces a positive current
\item $\nabla \times \vv{B} = \mu_o \vv{j}$ generates a positive B, opposing the change in flux
\begin{equation*}
\begin{aligned}
\dot{B} = \frac{\partial B}{\partial t}
\end{aligned}
\end{equation*}
\end{itemize}
We use this in Bdot probes for magnetic field measurement.
\begin{center}
\vspace{40mm}
\textbf{Figure 24}
\end{center}
\begin{framed}
\textbf{B-Dot Probe Theory}
A B-dot probe is used to measure the a time varying magnetic field produced by an electromagnetic waves propagating through a plasma. The term ?B-dot? comes from the mathematical notation $\dot{B} = \frac{\partial B}{\partial t}$ . These waves can be measured in situ by inserting the probe in a glass tube in the plasma chamber since glass is not electrically conductive.

Faraday?s law can be used to show that a time varying magnetic field can produce a voltage in a loop of wire. We can see this directly from the theory: The induced voltage is directly related to the flux through the loop:
\begin{equation*}
\begin{aligned}
V = - \frac{\mathrm{d} \Phi}{\mathrm{d}t} \qquad \qquad \text{(A)}
\end{aligned}
\end{equation*}
Where the flux is defined as $\Phi = a\, N \, B$. Here $a$ is the area of the loop and $N$ is the number of loops in the wire. In our case, there is only one loop in the probe, thus $N = 1$. Substituting the equation for the flux into Equation A, the result is an equation that directly relates the voltage to the change in magnetic field in the plasma.
\begin{equation*}
\begin{aligned}
V = - \frac{\mathrm{d} \, (a B)}{\mathrm{d}t} \qquad \qquad \text{(B)}
\end{aligned}
\end{equation*}
If the area of the loop is constant, then $a$ can be taken out of the time derivative from Equation B. Thus the time varying magnetic field is proportional to the amplitude of the induced voltage.
\begin{equation*}
\begin{aligned}
V = - a \,\frac{\mathrm{d} B}{\mathrm{d}t} \qquad \qquad \text{(C)}
\end{aligned}
\end{equation*}
\end{framed}
\end{document}