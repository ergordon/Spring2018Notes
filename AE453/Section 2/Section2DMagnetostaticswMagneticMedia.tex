\documentclass[11pt]{article}
\usepackage[utf8]{inputenc} % Para caracteres en espa�ol
\usepackage{amsmath,amsthm,amsfonts,amssymb,amscd}
\usepackage{multirow,booktabs}
\usepackage[table]{xcolor}
\usepackage{fullpage}
\usepackage{lastpage}
\usepackage{enumitem}
\usepackage{multicol}
\usepackage{fancyhdr}
\usepackage{mathrsfs}
\usepackage{wrapfig}
\usepackage{setspace}
\usepackage{esvect}
\usepackage{calc}
\usepackage{multicol}
\usepackage{cancel}
\usepackage{graphicx}
\graphicspath{ {pictures/} }
\usepackage[retainorgcmds]{IEEEtrantools}
\usepackage[margin=3cm]{geometry}
\usepackage{amsmath}
\newlength{\tabcont}
\setlength{\parindent}{0.0in}
\setlength{\parskip}{0.05in}
\usepackage{empheq}
\usepackage{framed}
\usepackage[most]{tcolorbox}
\usepackage{xcolor}
\colorlet{shadecolor}{orange!15}
\parindent 0in
\parskip 12pt
\geometry{margin=1in, headsep=0.25in}
\theoremstyle{definition}
\newtheorem{defn}{Definition}
\newtheorem{reg}{Rule}
\newtheorem{exer}{Exercise}
\newtheorem{note}{Note}
\newcommand{\volume}{{\ooalign{\hfil$V$\hfil\cr\kern0.08em--\hfil\cr}}}
\newcommand{\parr}{\mathbin{\|}} % Parralel Symbol
\begin{document}
\setcounter{section}{3}
\setcounter{page}{38}
\setcounter{equation}{74}
%\definecolor{babyblue}{rgb}{0.54, 0.81, 0.94}
\definecolor{babyblueeyes}{rgb}{0.63, 0.79, 0.95}
\definecolor{babyblue}{rgb}{0.69, 0.88, 0.9}

 \pagestyle{fancy}
\fancyhf{}
\rhead{Section 2: Electromagnetics}
\rfoot{Page \thepage}
\thispagestyle{empty}


\begin{center}
{\LARGE \bf Section 2: Electromagnetics}\\
{\large AE435}\\
Spring 2018
\end{center}
In this section, we will review the basics of charge, electricity, magnetism, and Maxwell equations.
\vspace{5mm}
\section{Magnetostatics with Magnetic Media}
\begin{center}
\vspace{25mm}
\end{center}
\tableofcontents
\newpage
\subsection{Effect of Magnetic Media}
In our previous discussion, we only considered magnetostatics involving steady currents in a vacuum. Now we will examine...
\begin{itemize}
\item \textbf{Question: }What happens if matter is present?
\item \textbf{Answer: }The magnetic field $\vv{B}$ changes!
\item \textbf{Reason: }Matter has micro-currents associated with the motion of the electrons around atoms, "atomic currents"
\item \textbf{Aftermath: }So now we must consider two kinds of currents:
\begin{itemize}
\item Conduction currents, involving free charges
\item Atomic currents, with no charge transport (to the first order)
\end{itemize}
\end{itemize}
\newline
Each atom has a magnetic dipole moment.
\begin{shaded}
 \newline
 \textbf{Magnetic Dipole Moment}
\begin{equation}
\begin{aligned}
\vv{m}_i = \frac{1}{2} J_i \oint_c \vv{r}_i \times \mathrm{d}\vv{l}
\end{aligned}
\end{equation}
\end{shaded}
We can define a macroscopic vector quantity analogous to polarization, known as the \textbf{magnetization} or the magnetic dipole moment per unit volume. 
\begin{shaded}
 \newline
 \textbf{Magnetization}
\begin{equation}
\begin{aligned}
\vv{M} = \lim_{\Delta\volume\rightarrow0} \, \frac{1}{\Delta\volume} \, \sum_i \vv{m}_i
\end{aligned}
\end{equation}
\end{shaded}
In the \textbf{unmagnetized state}, $\vv{M}=0$ because $\vv{m}_i$ have random orientations that cancel out. In the presence of an external $\vv{B}$, matter becomes organized and $\vv{M}$ can become nonzero depending on the material properties. 
\newpage
\begin{center}
\vspace{1mm}
\end{center}
\textbf{Magnetization Current: }How does magnetization give rise to currents?
\begin{center}
\vspace{40mm}
\textbf{Figure 20}
\end{center}

For a uniform $\vv{M}$, currents cancel in the interior but not on the exterior. The result is a net surface current as shown in Figure 20. 
\newline
\newline
Similarly, if $\vv{M}$ is non-uniform, we can have an internal net current.
\newline
\newline
\newline
We can define a \textbf{Magnetization Current Density: }
\newline
\begin{equation}
\begin{aligned}
\vv{j}_m = \nabla \times \vv{M}
\end{aligned}
\end{equation}
\newpage
\subsection{Total Magnetic Field}
To incorporate $\vv{j}_m$ into Ampere's Law (Equation 72) we have to modify the magnetic field equations, just as we modified Gauss' law to include $\rho_e$. 
\newline
\newline
As before, we still have no monopoles:
\newline
\begin{equation*}
\begin{aligned}
\nabla \cdot \vv{B} = 0
\end{aligned}
\end{equation*}
But now, Ampere's law becomes:
\newline
\begin{equation}
\begin{aligned}
\nabla \times \vv{B} = \mu_o \, (\vv{j}+\vv{j}_m)
\end{aligned}
\end{equation}
Using Equation 77, we can write this as:
\begin{equation*}
\begin{aligned}
\nabla \times \bigg(\frac{1}{\mu_o}\,\vv{B}-\vv{M}\bigg) = \vv{j}
\end{aligned}
\end{equation*}
Where $(\frac{1}{\mu_o}\,\vv{B}-\vv{M})$ depends only on conduction current density $\vv{j}$ as its source. As a results, we define a vector field:
\begin{shaded}
 \newline
 \textbf{Magnetic Intensity or "H"-field}
 \newline
\begin{equation}
\begin{aligned}
\vv{H} = \frac{1}{\mu_o}\,\vv{B}-\vv{M} \qquad \bigg[\frac{\text{A}}{m}\bigg] = [\text{Oersted}]
\end{aligned}
\end{equation}
\newline
\textbf{Note:} 1 $\frac{\text{A}}{m}$ = 0.01257 Oersted
\end{shaded}
Finally, \textbf{Ampere's Law for Magnetic Media} is:
\newline
\begin{equation}
\begin{aligned}
\nabla \times \vv{H} = \vv{j}
\end{aligned}
\end{equation}
\newpage
A comparison of Magnetostatics and Electrostatics:
\setlength{\arrayrulewidth}{.5 mm}
\setlength{\tabcolsep}{18pt}
\renewcommand{\arraystretch}{2}
\begin{center}
 \begin{tabular}{| c | c |} 
 \hline
 \rowcolor{gray} \textbf{Electrostatics} & \textbf{Magnetostatics} \\ [1ex] 
 \hline
 \rowcolor{lightgray} In vacuum (no $\rho_p$) & In vacuum (no $\vv{j}_m$) \\  [1ex] 
 \hline
 $\nabla \cdot \vv{E} = \frac{q}{\epsilon_o} \quad$ (isolated charges) & $\nabla \cdot \vv{B} = 0$ \\ [1ex] 
 $\nabla \cdot \vv{E} = \frac{\rho_e(\vv{r})}{\epsilon_o} \quad$ (distributed charges) & \\ [1ex] 
 $\nabla \times \vv{E} = 0 & $\nabla \times \vv{B} = \mu_o \vv{j}$ \\ [1ex] 
  \hline
  \rowcolor{lightgray} With media effects (finite $\rho_p$) & With media effects (finite $\vv{j}_m$) \\ [0.5ex] 
 \hline
 $\nabla \cdot \vv{E} = (\rho_f + \rhp_p)/\epsilon_o$ & $\nabla \cdot \vv{B} = 0$ \\ [1ex] 
 $\nabla \cdot \vv{D} = \rho_f$                                   &  \\ [1ex] 
 $\nabla \times \vv{E} = 0$                                        & $\nabla \times \vv{B} = \mu_o (\vv{j}+\vv{j}_m)$ \\ [1ex] 
 & $\nabla \times \vv{H} = \vv{j}$ \\ [1ex] 
 \hline
 \end{tabular}
\end{center}
We can also derive the intergral equation for magnetic intensity. From Equation 80, if we integrate over a surface element and apply Stokes theorem on a closed curve surrounding the surface, we get:
\begin{equation}
\begin{aligned}
\int_S \nabla \times \vv{H} \cdot \hat{n} \, \mathrm{d}A = \oint_C \vv{H} \cdot \mathrm{d} \vv{l} = \int_S \vv{j} \cdot \hat{n} \, \mathrm{d}A = J
\end{aligned}
\end{equation}
\textbf{Important Note:} This only applies for Magnetostatics. It does not work for time-varying fields.
\newpage
\subsection{Constitutive Equations/Relations}
\newpage
\subsection{Boundary Conditions}
\newpage
\subsection{Magnetic Flux}
\end{document}