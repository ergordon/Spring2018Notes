\documentclass[11pt]{article}
\usepackage[utf8]{inputenc} % Para caracteres en espa�ol
\usepackage{amsmath,amsthm,amsfonts,amssymb,amscd}
\usepackage{multirow,booktabs}
\usepackage[table]{xcolor}
\usepackage{fullpage}
\usepackage{lastpage}
\usepackage{enumitem}
\usepackage{multicol}
\usepackage{fancyhdr}
\usepackage{mathrsfs}
\usepackage{wrapfig}
\usepackage{setspace}
\usepackage{esvect}
\usepackage{calc}
\usepackage{multicol}
\usepackage{booktabs}% http://ctan.org/pkg/booktabs
\newcommand{\tabitem}{~~\llap{\textbullet}~~}
\usepackage{cancel}
\usepackage{graphicx}
\graphicspath{ {pictures/} }
\usepackage[retainorgcmds]{IEEEtrantools}
\usepackage[margin=3cm]{geometry}
\usepackage{amsmath}
\newlength{\tabcont}
\setlength{\parindent}{0.0in}
\setlength{\parskip}{0.05in}
\usepackage{empheq}
\usepackage{framed}
\usepackage[most]{tcolorbox}
\usepackage{xcolor}
\colorlet{shadecolor}{orange!15}
\parindent 0in
\parskip 12pt
\geometry{margin=1in, headsep=0.5in}
\theoremstyle{definition}
\newtheorem{defn}{Definition}
\newtheorem{reg}{Rule}
\newtheorem{exer}{Exercise}
\usepackage{setspace}
%\singlespacing
%\onehalfspacing
%\doublespacing
\setstretch{1.5}
\newtheorem{note}{Note}
\begin{document}
\setcounter{section}{0}
 \pagestyle{fancy}
\fancyhf{}
\rhead{AE370: Equation Sheet}
\begin{framed}
\textbf{Rectangle Rule for Integration} \newline
\begin{equation*}
\int_{x_{i-1}}^{x_i} f(x) \mathrm{d}x \approx \sum_{i=1}^{n} \, h \, f\bigg(\frac{x_{i-1}+x_{i}}{2}\bigg)
\end{equation*}
Where:
\begin{equation*}
\begin{split}
n &= \text{Number of Subintervals} \\
h &= \frac{x_{\text{final}} - x_{\text{initial}}}{n} \\
& = \text{Length of a Single Subinterval} \\
\end{split}
\end{equation*}
\end{framed}
\begin{framed}
\textbf{Trapezoidal Rule for Integration} \newline
\begin{equation*}
\int_{x_{i-1}}^{x_i} f(x) \mathrm{d}x \approx \sum_{i=1}^{n} \, \frac{h}{2} \, \bigg(f(x_{i-1})+f(x_{i})\bigg)
\end{equation*}
Where:
\begin{equation*}
\begin{split}
n &= \text{Number of Intervals} \\
h &= \frac{x_{\text{final}} - x_{\text{initial}}}{n} \\
& = \text{Length of a Single Subinterval}
\end{split}
\end{equation*}
\end{framed}
\begin{framed}
\textbf{Simpson Rule for Integration} \newline
\begin{equation*}
\int_{x_{i-1}}^{x_i} f(x) \mathrm{d}x \approx \frac{2}{3} \, I_{\text{rectangular}} + \frac{1}{3} \, I_{\text{trapezoidal}}
\end{equation*}
Where:
\begin{equation*}
\begin{split}
I_{\text{rectangular}} &= \text{Rectangle Rule Estimate} \\
I_{\text{trapezoidal}} &= \text{Trapezoidal Rule Estimate} \\
\end{split}
\end{equation*}
\end{framed}
\newpage
\begin{framed}
\textbf{Gauss Quadrature Rule for Integration} \newline
\begin{equation*}
\int_{-1}^{1} \, f_n (\xi) \, \mathrm{d}\xi = \sum_{k=1}^{q} f_n (\xi_k) \, w_k
\end{equation*}
\textbf{Basic Idea:} We will try to minimize the number of sampling points needed to integrate exactly a polynomial of a chosen degree n on the domain [-1, 1].
\newline
\newline
Where:
\newline
$w_k$ = Weighting Function
\begin{itemize}
\item \textbf{To integrate exactly (n=1) $f_1(\xi) = \alpha_0 + \alpha_1 \, \xi$} \\
\[ 
\left \{
  \begin{tabular}{cc}
  $w_1 = 2$ \\
  $\xi_1 = 0$
  \end{tabular}}
\]
\item \textbf{To integrate exactly (n=3) $f_3(\xi) = \alpha_0 + \alpha_1 \, \xi + \alpha_2 \, \xi^2 + \alpha_3 \, \xi^3$}\\
We need 2 sampling points (and 2 weights)
\[ 
\left \{
  \begin{tabular}{cc}
  $w_1 = 1$&$w_2 = 1$ \\
  $\xi_1 = -\frac{\sqrt{3}}{3}$ & $\xi_2 = \frac{\sqrt{3}}{3}$
  \end{tabular}}
\]
\item \textbf{To integrate exactly (n=5) $f_5(\xi) = \alpha_0 + \alpha_1 \, \xi + \alpha_2 \, \xi^2 + \alpha_3 \, \xi^3 + \alpha_4 \, \xi^4 + \alpha_5 \, \xi^5 $} \\
We need 3 sampling points (and 3 weights):
\[ 
\left \{
  \begin{tabular}{ccc}
  $w_1 = \frac{8}{9}$ & $w_2 = \frac{5}{9}$ & $w_3 = \frac{5}{9}$ \\
  $\xi_1 = 0$ & $\xi_2 = \sqrt{\frac{3}{5}}$ & $\xi_3 = -\sqrt{\frac{3}{5}}$
  \end{tabular}}
\]
\end{itemize}
\end{framed}

\newpage

\begin{framed}
\textbf{Coordinate Transformation}
\begin{equation*}
\begin{aligned}
\xi = \frac{2x}{b-a} + \frac{a+b}{a-b} = \frac{2x-(a+b)}{b-a}
\end{aligned}
\end{equation*}
such that
\begin{equation*}
\begin{aligned}
\int_{a}^{b}f(x)\, \mathrm{d}x = \frac{b-a}{2}\int_{-1}^{1} f\bigg(\frac{(b-a)\xi+(a+b)}{2}\bigg) \mathrm{d}\xi
\end{aligned}
\end{equation*}
\end{framed}

\begin{framed}
\textbf{Forward Difference Scheme}
\begin{equation*}
\begin{aligned}
f'(x) = \frac{f(x+h)-f(x)}{h} - \frac{h}{2} f''(\xi) \qquad \text{with } \xi \in (x,x+h)
\end{aligned}
\end{equation*}
\end{framed}

\begin{framed}
\textbf{Central Difference Scheme}
\begin{equation*}
\begin{aligned}
f'(x) = \frac{f(x+h)-f(x-h)}{2h} - \frac{h^2}{6} f'''(\xi) \qquad \text{with } \xi \in (x,x+h)
\end{aligned}
\end{equation*}
\end{framed}

\begin{framed}
\textbf{Richardson Extrapolation}
\begin{equation*}
\begin{aligned}
a_o = F(h) + \frac{F(h) - F(qh)}{q^p - 1} + O(h^r) \qquad \text{or} \qquad a_o = \frac{F(qh) - q^p\,F(h)}{1 - q^p}+ O(h^r)
\end{aligned}
\end{equation*}
Where:
\begin{equation*}
\begin{split}
p &= 1 \qquad \text{For Forward Difference Scheme since error is linear.} \\
p &= 2 \qquad \text{For Central Difference Scheme since error is squared.} \\
\end{split}
\end{equation*}

\end{framed}
\end{document}