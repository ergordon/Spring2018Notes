\documentclass[11pt]{article}
\usepackage[utf8]{inputenc} % Para caracteres en espa�ol
\usepackage{amsmath,amsthm,amsfonts,amssymb,amscd}
\usepackage{multirow,booktabs}
\usepackage[table]{xcolor}
\usepackage{fullpage}
\usepackage{lastpage}
\usepackage{enumitem}
\usepackage{multicol}
\usepackage{fancyhdr}
\usepackage{mathrsfs}
\usepackage{wrapfig}
\usepackage{setspace}
\usepackage{esvect}
\usepackage{calc}
\usepackage{multicol}
\usepackage{booktabs}% http://ctan.org/pkg/booktabs
\newcommand{\tabitem}{~~\llap{\textbullet}~~}
\usepackage{cancel}
\usepackage{graphicx}
\graphicspath{ {pictures/} }
\usepackage[retainorgcmds]{IEEEtrantools}
\usepackage[margin=3cm]{geometry}
\usepackage{amsmath}
\newlength{\tabcont}
\setlength{\parindent}{0.0in}
\setlength{\parskip}{0.05in}
\usepackage{empheq}
\usepackage{framed}
\usepackage[most]{tcolorbox}
\usepackage{xcolor}
\colorlet{shadecolor}{orange!15}
\parindent 0in
\parskip 12pt
\geometry{margin=1in, headsep=0.5in}
\theoremstyle{definition}
\newtheorem{defn}{Definition}
\newtheorem{reg}{Rule}
\newtheorem{exer}{Exercise}
\usepackage{setspace}
%\singlespacing
%\onehalfspacing
%\doublespacing
\setstretch{1.5}
\newtheorem{note}{Note}
\begin{document}
\setcounter{section}{0}
 \pagestyle{fancy}
\fancyhf{}
\rhead{AE370: Sample Problems}
{\large \bf Section 4.5 Character of Second Order PDE: Slide 37 Sample Problem}\\ \\
For the vairous Aerospace related PDE's; What is $(B^2 - 4AC)$? What type of PDE is it?
\newline
\noindent\rule{16.5cm}{0.4pt}
\begin{framed}
\textbf{Determining PDE Directions}
\begin{equation*}
\begin{aligned}
a \, \frac{\partial^2 T}{\partial x^2} + b \, \frac{\partial^2 T}{\partial x \, \partial y} + c \, \frac{\partial^2 T}{\partial y^2} + d \, \frac{\partial T}{\partial x} + e \, \frac{\partial T}{\partial y} + g \, T + h = 0
\end{aligned}
\end{equation*}
Such that the slope $(dx/dy)$ is controlled by the sign of $(b^2 - 4ac)$. In other words, If...
\begin{equation*}
\begin{aligned}
(b^2 - 4ac) < 0 &\rightarrow \text{the slope is imaginary (all directions)} \\
&\rightarrow \textbf{Elliptic PDE} \\
(b^2 - 4ac) = 0 &\rightarrow \text{There is only one slope (information uniformly in one direction)} \\
&\rightarrow \textbf{Parabolic PDE} \\
(b^2 - 4ac) > 0 &\rightarrow \text{There are two slopes (information in two paths)} \\
&\rightarrow \textbf{Hyperbolic PDE} \\
\end{aligned}
\end{equation*}
\end{framed}

\textbf{The Inviscid Stream-Function Equation:}
\begin{center}
\begin{tabular}{ c | c c c }
\begin{equation*}
\begin{aligned}
\frac{\partial^2 \psi }{\partial x^2} +\frac{\partial^2 \psi }{\partial y^2} = 0
\end{aligned}
\end{equation*} &
\begin{equation*}
\begin{aligned}
A & = 1\\
B & = 0\\
C & = 1\\
\end{aligned}
\end{equation*} & \begin{equation*}
\begin{aligned}
(B^2 - 4AC) = -4
\end{aligned}
\end{equation*} & The PDE is Ellipitic \\ 
\end{tabular}
\end{center}

\textbf{The Inviscid Linearized 2-D Compressible Flow Equation:}
\begin{center}
\begin{tabular}{ c | c c c }
\begin{equation*}
\begin{aligned}
(M_o - 1) \, \frac{\partial^2 \phi }{\partial x^2} -\frac{\partial^2 \phi }{\partial y^2} = 0
\end{aligned}
\end{equation*} &
\begin{equation*}
\begin{aligned}
A & = 1\\
B & = 0\\
C & = 1\\
\end{aligned}
\end{equation*} & \begin{equation*}
\begin{aligned}
(B^2 - 4AC) > 0
\end{aligned}
\end{equation*} & The PDE is Hyperbolic \\ 
\end{tabular}
\end{center}
\newpage

\textbf{The Beam Torsion Equation:}
\begin{center}
\begin{tabular}{ c | c c c }
\begin{equation*}
\begin{aligned}
\frac{\partial^2 \psi }{\partial x^2} +\frac{\partial^2 \psi }{\partial y^2} = -2 \, \mu \, \Theta
\end{aligned}
\end{equation*} &
\begin{equation*}
\begin{aligned}
A & = (M_o - 1)\\
B & = 0\\
C & = 1\\
H &= 2 \, \mu \, \Theta
\end{aligned}
\end{equation*} & \begin{equation*}
\begin{aligned}
(B^2 - 4AC) -4
\end{aligned}
\end{equation*} & The PDE is Elliptic \\ 
\end{tabular}
\end{center}

\textbf{The Beam Wave Equation:}
\begin{center}
\begin{tabular}{ c | c c c }
\begin{equation*}
\begin{aligned}
\frac{\partial^2 u}{\partial t^2} = \frac{E}{\rho} \, \frac{\partial^2 u }{\partial x^2}
\end{aligned}
\end{equation*} &
\begin{equation*}
\begin{aligned}
A & = \frac{E}{\rho} \\
B & = 0\\
C & = -1\\
\end{aligned}
\end{equation*} & \begin{equation*}
\begin{aligned}
(B^2 - 4AC) > 0
\end{aligned}
\end{equation*} & The PDE is Hyperbolic 
\end{tabular}
\end{center}

\textbf{The Thermal 1-D Heat Diffusion Equation:}
\begin{center}
\begin{tabular}{ c | c c c }
\begin{equation*}
\begin{aligned}
\frac{\partial^2 T}{\partial x^2} = \frac{\rho \, c}{k} \, \frac{\partial T }{\partial t}
\end{aligned}
\end{equation*} &
\begin{equation*}
\begin{aligned}
A & = 1 \\
B & = 0\\
C & = 0\\
\end{aligned}
\end{equation*} & \begin{equation*}
\begin{aligned}
(B^2 - 4AC) = 0
\end{aligned}
\end{equation*} & The PDE is Parabolic 
\end{tabular}
\end{center}

\textbf{The Laminar Boundary Layer Equation:}
\begin{center}
\begin{tabular}{ c | c c c }
\begin{equation*}
\begin{aligned}
u \, \frac{\partial u}{\partial x} + v \, \frac{\partial u}{\partial y} = v \, \frac{\partial^2 u }{\partial y^2}
\end{aligned}
\end{equation*} &
\begin{equation*}
\begin{aligned}
A & = 0 \\
B & = 0\\
C & = v\end{aligned}
\end{equation*} & \begin{equation*}
\begin{aligned}
(B^2 - 4AC) = 0
\end{aligned}
\end{equation*} & The PDE is Parabolic 
\end{tabular}
\end{center}
\newpage
{\large \bf Sample Problem}\\ \\
How do we discretize a partial function of two variables?
 \begin{equation*}
\begin{aligned}
\frac{\partial f}{\partial x \partial y}\bigg|_{i,j}
\end{aligned}
\end{equation*}
\noindent\rule{16.5cm}{0.4pt}

Let's Expand this a bit
 \begin{equation}
\begin{aligned}
\frac{\partial f}{\partial x \partial y}\bigg|_{i,j} = \frac{\partial f}{\partial x}\bigg(\frac{\partial f}{\partial y}\bigg)
\end{aligned}
\end{equation}
What is the discretized form of $\frac{\partial f}{\partial y}$ ? Well...
 \begin{equation}
\begin{aligned}
\frac{\partial f}{\partial y}\bigg|_{i,j} = \frac{f_{i,j+1}-f_{i,j-1}}{2 \, \Delta y}
\end{aligned}
\end{equation}
Substituting this Equation 2 into Equation 1, we get....

 \begin{equation*}
\begin{aligned}
\frac{\partial f}{\partial x}\bigg(\frac{\partial f}{\partial y}\bigg) &= \frac{\partial f}{\partial x}\Bigg(\frac{f_{i,j+1}- \,f_{i,j-1}}{2 \, \Delta y}\Bigg) \\ \\ 
& = \frac{\frac{\partial f}{\partial y}\bigg|_{i+1,j} - \frac{\partial f}{\partial y}\bigg|_{i-1,j}}{2 \, \Delta y} \\ \\
& ... \\ \\
&= \frac{f_{i+1,j+1} -f_{i+1,j-1}  -f_{i-1,j+1}  + f_{i-1,j-1} }{4 \, \Delta x \, \Delta y}
\end{aligned}
\end{equation*}
\end{document}