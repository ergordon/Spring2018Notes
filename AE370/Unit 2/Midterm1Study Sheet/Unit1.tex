\documentclass[11pt]{article}
\usepackage[utf8]{inputenc} % Para caracteres en espa�ol
\usepackage{amsmath,amsthm,amsfonts,amssymb,amscd}
\usepackage{multirow,booktabs}
\usepackage[table]{xcolor}
\usepackage{fullpage}
\usepackage{lastpage}
\usepackage{enumitem}
\usepackage{multicol}
\usepackage{fancyhdr}
\usepackage{mathrsfs}
\usepackage{wrapfig}
\usepackage{setspace}
\usepackage{esvect}
\usepackage{calc}
\usepackage{multicol}
\usepackage{booktabs}% http://ctan.org/pkg/booktabs
% Two more packages that make it easy to show MATLAB code
\usepackage[T1]{fontenc}
\usepackage[framed]{matlab-prettifier}
\lstset{
	style = Matlab-editor,
	basicstyle=\mlttfamily\small,
}

\newcommand{\tabitem}{~~\llap{\textbullet}~~}
\usepackage{cancel}
\usepackage{graphicx}
\graphicspath{ {pictures/} }
\usepackage[retainorgcmds]{IEEEtrantools}
\usepackage[margin=3cm]{geometry}
\usepackage{amsmath}
\newlength{\tabcont}
\setlength{\parindent}{0.0in}
\setlength{\parskip}{0.05in}
\usepackage{empheq}
\usepackage{framed}
\usepackage[most]{tcolorbox}
\usepackage{xcolor}
\colorlet{shadecolor}{orange!15}
\parindent 0in
\parskip 12pt
\geometry{margin=1in, headsep=0.5in}
\theoremstyle{definition}
\newtheorem{defn}{Definition}
\newtheorem{reg}{Rule}
\newtheorem{exer}{Exercise}
\usepackage{setspace}
%\singlespacing
%\onehalfspacing
%\doublespacing
\setstretch{1.5}
\newtheorem{note}{Note}
\begin{document}
\setcounter{section}{0}
 \pagestyle{fancy}
\fancyhf{}
\rhead{AE370: Equation Sheet}

\begin{framed}
\textbf{Determining PDE Directions}
\begin{equation*}
\begin{aligned}
a \, \frac{\partial^2 T}{\partial x^2} + b \, \frac{\partial^2 T}{\partial x \, \partial y} + c \, \frac{\partial^2 T}{\partial y^2} + d \, \frac{\partial T}{\partial x} + e \, \frac{\partial T}{\partial y} + g \, T + h = 0
\end{aligned}
\end{equation*}

Such that the slope $(dx/dy)$ is controlled by the sign of $(b^2 - 4ac)$. In other words, If...
\begin{equation*}
\begin{aligned}
(b^2 - 4ac) < 0 &\rightarrow \text{the slope is imaginary (all directions)} \\
&\rightarrow \textbf{Elliptic PDE} \\ \\
(b^2 - 4ac) = 0 &\rightarrow \text{There is only one slope (information uniformly in one direction)} \\
&\rightarrow \textbf{Parabolic PDE} \\ \\
(b^2 - 4ac) > 0 &\rightarrow \text{There are two slopes (information in two paths)} \\
&\rightarrow \textbf{Hyperbolic PDE} \\ \\
\end{aligned}
\end{equation*}
\end{framed}

\begin{framed}
\textbf{First Order Forward Difference Scheme}
\begin{equation*}
\begin{aligned}
\frac{\partial f}{\partial x}\bigg|_{i} &= \frac{f_{+1}-f_{i}}{\Delta x} + O(\Delta x)
\end{aligned}
\end{equation*}
First order because of the error. Forward Difference because of the index.
\end{framed}

\begin{framed}
\textbf{First Order Backward Difference Scheme}
\begin{equation*}
\begin{aligned}
\frac{\partial f}{\partial x}\bigg|_{i} &= \frac{f_i - f_{i-1}}{\Delta x} + O(\Delta x)
\end{aligned}
\end{equation*}
\end{framed}

\begin{framed}
\textbf{Second Order Backward Difference Scheme}
\begin{equation*}
\begin{aligned}
\frac{\partial f}{\partial x}\bigg|_{i} &= \frac{3\,f_i - 4\,f_{i-1} + f_{i-2}}{2 \, \Delta x} + O(\Delta x^2)
\end{aligned}
\end{equation*}
\end{framed}

\begin{framed}
\textbf{Second Order Central Difference Scheme for the First Derivative}
\begin{equation*}
\begin{aligned}
\frac{\partial f}{\partial x}\bigg|_{i} &= \frac{f_{i+1} - f_{i-1}}{2 \, \Delta x} + O(\Delta x^2)
\end{aligned}
\end{equation*}
\end{framed}
\begin{framed}
\textbf{Second Order Central Difference Scheme for the Second Derivative}
\begin{equation*}
\begin{aligned}
\frac{\partial^2 f}{\partial x^2}\bigg|_{i} = \frac{f_{i+1} - 2 \, f_i + f_{i-1}}{\Delta x^2} + O(\Delta x^2)
\end{aligned}
\end{equation*}
\end{framed}

\begin{framed}
\textbf{Types of Boundary Conditions}
\begin{equation*}
\begin{aligned}
\textbf{Dirichlet} \text{ (fixed):} &\qquad& f&=C_1\\ \\
\textbf{Neumann} \text{ (derivative):} &\qquad& \frac{\partial f}{\partial x}&=C_2\\ \\
\textbf{Cauchy} \text{ (mixed):} &\qquad& f + C_1 \,\frac{\partial f}{\partial x} &=C_2\\
\end{aligned}
\end{equation*}
\end{framed}
\newpage
\begin{framed}
\textbf{Derivative Boundary Conditions with Ghost Cell Approach}

Recall Neumann BC:
\begin{equation*}
\begin{aligned}
\frac{\partial f}{\partial x}=C_1
\end{aligned}
\end{equation*}
Recall Second Order Central Difference:
\begin{equation*}
\begin{aligned}
\frac{\partial f}{\partial x}\bigg|_{i} = \frac{f_{i+1} - f_{i-1}}{2 \, \Delta x} + O(\Delta x)
\end{aligned}
\end{equation*}
Equating these and equating for the ghost cell ($f_{n+1}$) we get:
\begin{equation*}
\begin{aligned}
[f_{i+1} - f_{i-1}] = 2 \, C_1 \, \Delta x \quad \rightarrow \quad f_{i+1} = 2 \, C_1 \, \Delta x + f_{i-1}
\end{aligned}
\end{equation*}
Substitute the ghost term into the GDE (thereby eliminating the ghost cell ($f_{n+1}$)). The resulting equation is the boundary condition equation ready for implementation.
\end{framed}

\end{document}